\documentclass[12pt]{scrartcl} % Beginn der LaTeX-Datei

% Größerer Zeilenabstand in mehrzeiligen amsmath-Umgebungen
\addtolength{\jot}{0.2em}

% Schicke Nummerierung
\usepackage{enumitem}
% Kleinere itemize-Bulletpoints
\newcommand{\sbt}{\,\begin{picture}(-1,1)(-1,-3)\circle*{3}\end{picture}\ }
\renewcommand{\labelitemi}{\sbt}

% Fußnoten von Exponenten unterscheiden
\renewcommand{\thefootnote}{[\arabic{footnote}]}

% Erleichtert Mathe bzw. Formelsatz
\usepackage{mathtools, amsmath, amssymb, amsthm}
\usepackage{nicefrac}

% Commands and operators
\DeclareMathOperator{\powerset}{\mathcal{P}}
\DeclareMathOperator{\sym}{Sym}
\DeclareMathOperator{\GL}{GL}
\DeclareMathOperator{\SL}{SL}
\DeclareMathOperator{\abb}{Abb}
\DeclareMathOperator{\id}{id}
\DeclareMathOperator{\ord}{ord}
\DeclareMathOperator{\im}{im}
\DeclareMathOperator{\sgn}{sgn}
\DeclareMathOperator{\supp}{supp}
\DeclareMathOperator{\Aut}{Aut}
\DeclareMathOperator{\rank}{Rang}

% Further shorthands for maths
\newcommand{\inv}[1]{\left(#1\right)^{-1}}
\newcommand{\Inv}[1]{#1^{-1}}
\newcommand{\divides}{\mathrel{\bigm|}}

% Common groups and fields
\newcommand\N{\mathbb{N}}
\newcommand\Z{\mathbb{Z}}
\newcommand\Q{\mathbb{Q}}
\newcommand\R{\mathbb{R}}
\newcommand\C{\mathbb{C}}

% Environments
% Stuff with proof
\newtheorem{thm}{Satz}[section]
\newtheorem{lemma}[thm]{Lemma}
\newtheorem{kor}[thm]{Korollar}
% Definition-like stuff
\theoremstyle{definition}
\newtheorem*{defn}{Definition}
\newtheorem{ex}{Beispiel}
% Remark-like stuff
\theoremstyle{remark}
\newtheorem*{notation}{Notation}
\newtheorem*{nb}{Bemerkung}

% Graphiken
% Für Graphik-Einbindung
\usepackage{graphicx}
\usepackage{float} 
%\usepackage[dvips]{hyperref}
\usepackage{tikz-cd}
\tikzset{ % das braucht man für die kommutativen Diagramme, wenn man babel german benutzt!!
	every picture/.append style={
		execute at begin picture={\shorthandoff{"}},
		execute at end picture={\shorthandon{"}}
	}
}

% Einstellungen, wenn man deutsch schreiben will
% Trennregeln etc.
\usepackage[ngerman]{babel}
% Automatisch korrekte Anführungszeichen
\usepackage[autostyle=true, german=quotes]{csquotes}
\MakeOuterQuote{"}
% Damit UTF-8 funktoniert
\usepackage[utf8]{inputenc}
%\usepackage[T1]{fontenc}

% Um neu definierte Bezeichnungen hervorzuheben
% ZZ haben wir noch einen Mix aus \underline und \emph
\newcommand{\defi}{\underline}

% TODO Bspw. ord wird zZ als drei Variablen (o, r, d) gesetzt. Zu einem vernünftigen Operator ändern?
	% Ich fang einfach mal damit an. -- Peter
% TODO Einige Beweise sind schrecklich strukturiert und zum Teil rettbar fehlerhaft. Darf ich?
% TODO einheitliche subset-Notation?
	% Ich wäre für \subseteq und \subsetneq um Missverständnisse auszuschließen. -- Peter
% TODO emph statt underline?
	% Wär ich für. -- Peter
% TODO KatTheo-Notation für Funktionen oder nicht? (\xrightarrow{f})
% TODO Wir sollten mal über den massiven \\-Spam reden. Ich habe bisher eine sinnvolle Verwendung und ca. 10^4 unsinnige gesehen.
% TODO Die Nummerierungen sind inkonsitent. Spricht was dagegen, sich in der Präambel auf was zu einigen und die [label=(\roman*)] etc. rauszukanten?
        % Ich wäre dafür Lemmata, Sätze und Korollare gemeinsam und zudem kapitelweise zu nummerieren. -- Fabian
% TODO p-Primäre Komponenten Notation: p als Subindex?
% TODO Einheitliche Verwendung des Begriffs nicht-leer/nicht leer/nichtleer
% TODO Menge der Nebenklassen Einheitlich notieren (mit oder ohne Verwendung von nicefrac)

\begin{document}

\author{Sebastian Bechtel, Isburg Knof, Theresa Tran} % FIXME Flo, Philipp und Fabian?
\title{Einführung in die Algebra}
%\date weglassen -> Automatisch aktuelles Datum

\maketitle

\section{Gruppen}

\begin{defn}
	Eine (innere) \emph{Verknüpfung} auf einer Menge $M\neq \emptyset$ ist eine Abbildung $M\times M\to M, (a,b)\mapsto a\cdot b$.
\end{defn}

\begin{defn}
	Eine \emph{Gruppe} ist eine Menge $G\neq \emptyset$ zusammen mit einer Verknüpfung $\cdot$, sodass Assoziativität (A), Existenz eines neutralen Elements (N) und Existenz inverser Elemente (I) erfüllt sind.
	$G$ ist \emph{abelsch}, falls Kommutativität (K) gilt.
\end{defn}

\begin{ex}
	\begin{enumerate}
	\item $\mathbb{Z}, \mathbb{Q}, \mathbb{R}, \mathbb{C}$ sind abelsche Gruppen mit $+$ als Verknüpfung.
	\item $\mathbb{Q}^*=\mathbb{Q}\setminus \{0\}, \mathbb{R}^*, \mathbb{C}^*$ mit Multiplikation sind abelsche Gruppen.
	\item Für eine Menge $M$ ist $\sym(M)$ ist eine Gruppe, aber für $|M| > 2$ nicht abelsch.
	\end{enumerate}
\end{ex}

\begin{lemma}
	\begin{enumerate}[label=\alph*)]
	\item Das neutrale Element ist eindeutig.
	\item Inverse Elemente sind eindeutig.
	\end{enumerate}
\end{lemma}

\begin{proof}
	\begin{enumerate}[label=\alph*)]
	\item Seien $e,f$ neutrale Elemente, dann gilt $e=ef=f$.
	\item Sei $a\in G$ und $b,b'\in G$ inverse Elemente. Dann gilt $b'=b'e=b'(ab)=(b'a)b=eb=b$.
	\end{enumerate}
\end{proof}

\begin{notation}
	Multiplikativ: $a\cdot b$ oder $ab$, neutrales Element $e$ oder $1$, inverses Element von $a\in G$ ist $\Inv a$.
\end{notation}

\begin{lemma}
	Es sei $\mathcal{G}=(G,\cdot)$ eine Menge mit assoziativer Verknüpfung, einem linksneutralen Element und linksinversen Elementen, dann ist $\mathcal{G}$ eine Gruppe.
\end{lemma}

\begin{proof}
	Sei $a\in G$ und $b\in G$ das dazugehörige Linksinverse mit $ba=e$.
	Nun gibt es auch $c\in G$ mit $cb=e$.
	Also gilt $ab=eab=cbab=ceb=cb=e$ und \(b = a^{-1}\) ist auch rechtsinvers zu \(a\).

	Nun gilt für alle $a\in G$, dass $ae=a(\Inv aa)=ea=a$ ist.
	Somit ist \(e\) auch rechstneutral und die Gruppenaxiome gelten.
\end{proof}

\begin{lemma}
	\begin{enumerate}
	\item $\inv{\Inv{a}}=a$, $\inv{ab}=\Inv b\Inv a$ % TODO Eigentlich braucht man für beides hier die Eindeutigkeit, welche unten gezeigt wird. Umsortieren?
	\item $ab=ac$ impliziert $b=c$ für alle $a,b,c\in G$.
	\item Für $a,b\in G$ gibt es genau ein $x\in G$, sodass $ax=b$.
	\end{enumerate}
\end{lemma}

\begin{proof}
	\begin{enumerate}
	\item $\inv{\Inv a}=a$ klar. Für $a,b\in G$: $(\Inv b\Inv a)ab=\Inv b(\Inv aa)b=\Inv beb=\Inv bb=e$ (andere Richtung analog) % FIXME Welche andere Richtung? -- Gibt keine andere (Fabian)
	\item $ab=ac$ impliziert $\Inv a(ab)=\Inv a(ac)$ impliziert $b=c$
	\item Setze $x=\Inv ab$, dann erhält man $ax=a(\Inv ab)=(a\Inv a)b=eb=b$.
		Die Eindeutigkeit folgt aus Punkt 2. \qedhere
	\end{enumerate}
\end{proof}

\begin{defn}
	Sei $a\in G$, $(G,\cdot)$ Gruppe.
	Für $n\in \mathbb{Z}$ definiere:
	\begin{align*}
		a^0 &:= e\\
		a^n &:= a^{n-1}a \quad \text{ für \(n\geq 1\)}\\
		a^n &:= \left(\Inv a\right)^{-n} \quad \text{ für \(n < 0\)}
	\end{align*}
\end{defn}

\begin{lemma}
	Für $a\in G$ und \(n, m \in\mathbb Z\) gelten $a^n a^m=a^{n+m}=a^m a^n$ und $\left(a^m \right)^n = a^{n\cdot m}$.
	Falls \(b \in G\) mit \(a\) kommutiert, also $ab=ba$ ist, gilt $\left(ab \right)^n = a^n b^n$.
\end{lemma}

\begin{ex}
	\begin{enumerate}
	\item $K$ Körper, dann ist $\GL_n(K)$ eine Gruppe bzgl. Matrixmultiplikation.
	\item $M\neq \emptyset$ Menge, $(G, \cdot)$ Gruppe, definiere $\abb(M,G):=G^M$.
		Für $f,g\in \abb(M,G)$ ist $f\cdot g$ gegeben durch $(f\cdot g)(m)=f(m)\cdot g(m)$ für $m\in M$.
		Dann ist $(\abb(M,G), \cdot)$ eine Gruppe.
	\end{enumerate}
\end{ex}

\section{Untergruppen}

\begin{defn}
	Sei $(G, \cdot)$ Gruppe.
	Eine Teilmenge $H\subset G$ heißt Untergruppe von $G$, falls $(H, \cdot)$ eine Gruppe ist.

	Äquivalent dazu:
	\begin{enumerate}[label=(\roman*)]
	\item Für $a,b\in H$ gilt $ab\in H$ (Abgeschlossenheit)
	\item $e\in H$
	\item Für $a\in H$ ist $\Inv a \in H$
	\end{enumerate}
\end{defn}

\begin{nb}
	Wenn (i) und (iii) gelten, folgt (ii) bereits aus \(H \neq \emptyset\).
\end{nb}

\begin{thm}
	Sei $(G, \cdot)$ Gruppe und $H\subset G$ nicht-leer.
	Dann ist $H$ eine Untergruppe von $(G, \cdot)$, gdw. $a\Inv b\in H$ für alle $a,b\in H$.
\end{thm}

\begin{proof}
	"$\Rightarrow$" \checkmark

	"$\Leftarrow$"
	\begin{itemize}
	\item Da \(H\) nicht leer ist, gibt es ein \(a \in H\).
		Somit ist auch $e = aa^{-1}\in H$.
	\item Für beliebige \(a \in H\) muss also auch \(a^{-1} = ea^{-1} \in H\) sein.
	\item Schließlich erhalten wir, dass für beliebige \(a, b \in H\) auch \(b^{-1} \in H\) und somit \(ab = a(b^{-1})^{-1} \in H\) ist. \qedhere
	\end{itemize}
\end{proof}

\begin{ex}
	\begin{enumerate}[label=(\alph*)]
	\item Für alle Gruppen $(G,\cdot)$ sind $\{e\}$ und $G$ jeweils Untergruppen, die sogenannten \emph{trivialen Untergruppen}.
	\item Für beliebige Körper $K$ induziert $\SL_n(K)=\{A\in \GL_n(K): \det(A)=1\}$ eine Untergruppe von $\GL_n(K)$, die spezielle lineare Gruppe.
	\end{enumerate}
\end{ex}

\begin{defn}
	Eine Untergruppe heißt \emph{echt}, falls sie nicht trivial ist.
\end{defn}

\begin{lemma}
	Es sei $(H_{j})_{j \in J}$ eine Familie von Untergruppen $H_{j} \subset G$.
	Dann ist $\bigcap_{j \in J}H_{j}$ eine Untergruppe von G.
\end{lemma}

\begin{proof}
	Die Charakterisierung aus dem letzten Satz bleibt trivial unter Schnitten erhalten.
\end{proof}

\begin{defn}
	Es sei $M$ eine Teilmenge von $G$.
	Die \emph{von $M$ erzeugte Untergruppe} ist der Durchschnitt aller Untergruppen, die $M$ enthalten.
\end{defn}

\begin{notation}
	$\langle M \rangle =\bigcap_{M \subset H \subset G}H$, wobei $H$ Untergruppe
\end{notation}

\begin{nb}
	\begin{enumerate}[label=(\alph*)]
	\item $\langle \emptyset \rangle = \lbrace e \rbrace$
	\item Für $M \neq \emptyset$ gilt: $\langle M \rangle = \lbrace m_{1}^{\varepsilon_{1}} \cdot \dots \cdot m_{n}^{\varepsilon_{n}} : m_{1},\dots,m_{n} \in M, \varepsilon_{1},\dots,\varepsilon_{n} \in \lbrace -1,+1\rbrace, n \geq 0 \rbrace$
	\item Für $M = \lbrace g \rbrace$ gilt: $\langle g \rangle = \lbrace g^{n} : n \in \mathbb{Z} \rbrace$. Diese Untergruppe wird die von $g$ erzeugte zyklische Untergruppe von $G$ genannt.
	\end{enumerate}
\end{nb}

\begin{defn}
	Eine Gruppe $G$ heißt \emph{zyklisch}, falls $G = \langle g \rangle$ für ein $g \in G$ gilt. \newline
	Ist $G = \langle M \rangle$ mit $M$ endlich, so heißt $G$ \emph{endlich erzeugt}.
\end{defn}

\begin{defn}
	\begin{enumerate}[label=(\roman*)]
	\item Die \emph{Ordnung einer Gruppe} $G$ ist $\ord(G):=\vert G \vert$.
	\item Die \emph{Ordnung eines Elements} $g \in G$ ist $\ord(g):=\ord(\langle g \rangle)$.
	\item Ist $\ord(g)$ endlich, dann hat g \emph{endliche Ordnung}.
	\end{enumerate}
\end{defn}

\begin{notation}
	$(n,s)$ bezeichnet den größten gemeinsamen Teiler.
\end{notation}

\begin{thm}
	Sei $G$ Gruppe, $g \in G$
	\begin{enumerate}
	\item Falls \(g\) nicht endliche Ordnung hat, sind alle Potenzen von \(g\) verschieden
	\item Falls \(g\) endliche Ordnung hat, gibt es ein \(m > 0\) mit \(g^m = e\).
		Mit \(n := \ord(g)\) gelten
		\begin{enumerate}[label=(\alph*)]
		\item $n = \ord(g) = \min \lbrace m>0 : g^{m}=e \rbrace$
		\item $g^{m}=e \Longleftrightarrow m=nk$ für ein $k \in \mathbb{Z}$
		\item $\langle g \rangle = \lbrace e, g^{1},\dots,g^{n-1}\rbrace$
		\end{enumerate}
	\item Falls \(g\) endliche Ordnung $n := \ord(g)$ hat, gilt $\ord(g^{s}) = \frac{n}{(n,s)}$ für alle \(s \in \mathbb N\).
	\end{enumerate}
\end{thm}

\begin{proof} % TODO Diesen Beweis überspringe ich gerade. -- Peter 
	\begin{enumerate}
		\item Wir nehmen an: Für $i,j \in \mathbb{Z}$, o.B.d.A. $j>i$ gilt $g^{i}=g^{j}$.
		Dann gilt $g^{j-i}=g^{j}(g^{i})^{-1}=e$.
		Es sei dann n die kleinste positive Zahl, die $g^{n}=e$ erfüllt.
		Sei $m \in \mathbb{Z}$ beliebig.
		Der Divisionsalgorithmus liefert: $m=kn+r$ für $0 \leq r < n$ und $k,r \in \mathbb{Z}$.

		Dann gilt:
		$g^{m}=g^{kn+r}=g^{kn}g^{r}=(g^{n})^{k}g^{r}=eg^{r}=g^{r}$.
		Daraus folgt $\langle g \rangle = \lbrace g^{m} : m \in \mathbb{Z}\rbrace = \lbrace g^{r} : r=0,\dots,n-1\rbrace$.
		Besonders gilt $\ord(g)=n$ ist endlich.
		$\lceil$ Dies zeigt $\Rightarrow$, $\Leftarrow$ klar, dann ist $\langle g \rangle = \lbrace g^{m} : m \in \mathbb{Z}\rbrace$ unendlich $\rfloor$
		\item Alle $g^{r}$ mit $0 \leq r \leq n-1$ sind verschieden, da:
		$g^{i}=g^{j} \Rightarrow g^{j-i}=e \Rightarrow j-i = kn$ mit $k\in\mathbb{Z} \Rightarrow i \equiv j \pmod{n} \Rightarrow i=j$ falls $0\leq i,j\leq n-1$.
		Dies liefert $g^{r}$ mit $0 \leq r \leq n-1$ sind paarweise verschieden und es gilt: $\ord(g)=n$. $\lceil a$ und $c\rfloor$
		Aus dem Divisionsalgorithmus folgt (b): $g^{m}=e \Leftrightarrow e=g^{kn+r}=g^{r}$ mit $m = kn+r, 0 \leq r < n \Leftrightarrow r=0$.
		Also $m=kn$ mit $k \in \mathbb{Z}$. % FIXME Scoping-Katastrophe.
		\item Es sei $m=\ord(g^{s}), n =\ord(g)$.
		Aus $(g^{s})^{m}=e$ folgt (siehe 2), dass $n$ ein Teiler von $sm$ ist.
		Dies liefert: $\frac{n}{(s,n)} \divides \frac{s}{(s,n)}m$. Somit $\frac{n}{(s,n)} \divides m$.
		Nun möchten wir noch zeigen: $m \divides \frac{n}{(s,n)}$. $(g^{s})^{\frac{n}{(s,n)}} = (g^{n})^{\frac{s}{(s,n)}}=e^{\frac{s}{(s,n)}}=e$.
		Daraus folgt $m \divides \frac{n}{(s,n)}$ (wegen 2).
		Also gilt $m = \frac{n}{(s,n)}$. \qedhere
	\end{enumerate}
\end{proof}

\begin{lemma}
        Es sei $G = \langle g \rangle$ eine zyklische Gruppe und $H \subset G,\; H\neq\{e\}$ eine Untergruppe von $G$. % TODO Wieso nicht {e}? Da gilt die Aussage trivial mit m = 0. Vorschlag: \mathbb{N} -> \mathbb{N}_0. Dann muss im proof aber noch eine Fallunterscheidung gemacht werden H = {e} und H \neq {e}.
	Dann gibt es ein $m\in\mathbb{N}$ mit $H=\langle g^m\rangle$.
\end{lemma}

\begin{proof}
	Wir setzen: $m := \min\lbrace k>0 : g^{k} \in H \rbrace$.
	Die betrachtete Menge ist dabei nicht-leer, denn ist $h \in H, h \neq e$, dann gilt $h = g^{k}$ für ein $k\in\mathbb{Z}$, da $g$ Erzeuger von $G$ ist.
	Ist $k>0$, so ist $k$ in der Menge.
	Ist $k<0$, so gilt $g^{-k}=\inv{{g^k}}=\Inv{h}\in H$ und damit ist $-k$ in der Menge enthalten.
	Wir wollen nun zeigen: $\langle g^{m} \rangle = H$.
	\begin{enumerate}
	\item $\langle g^{m} \rangle \subset H$ gilt wegen $g^{m} \in H$.
	\item Es sei $j \in \mathbb{Z}$ mit $g^{j} \in H$. Divisionsalgorithmus liefert $j=lm+r$ mit $0 \leq r < m$: $g^{j} \in H \Rightarrow g^{r}=g^{-lm}g^{lm+r}=(g^{m})^{-l}g^{j}$. Also $g^{r} \in H$. Aus der Minimalität von M folgt $r=0$. Dies liefert $g^{j}=(g^{m})^{l} \in \langle g^{m} \rangle$ und somit gilt: $H \subset \langle g^{m} \rangle$ und die zwei Untergruppen stimmen überein. \qedhere
	\end{enumerate}
\end{proof}

Ähnlich kann man zeigen:

\begin{thm} % FIXME Beweis? -- Fixed? (Fabian)
	Alle Untergruppen einer zyklischen Gruppe sind zyklisch.
        %TODO \ord(G) -> \lvert G \lvert, \ord nur für Gruppenelemente verwenden
	Ist $\ord(G)=n$ endlich und $m$ ein Teiler von $n$, so ist $H = \langle g^{\frac{n}{m}}\rangle$ die einzige Untergruppe der Ordnung $m$.
\end{thm}

\begin{proof}
	Dass jede Untergruppe $U$ einer zyklischen Gruppe $G = \langle g \rangle$ wiederum zyklisch ist, folgt aus dem vorangehenden Beweis. 

	Sei nun $\lvert G \rvert = \ord(g) = n < \infty$.
	Des Weiteren sei $m \in \mathbb{N}$ mit $m \divides n$ und $U$ eine Untergruppe von $G$ mit $\lvert U \rvert = m$.
	Da $U$ zyklisch ist, existiert ein $k \in \mathbb{N}$, sodass $U = \langle g^k \rangle$.
	Mit dem Satz von Lagrange (später) folgt $(g^k)^m = e = g^n$, also $n \divides km$.
	Daraus folgt $\frac{n}{m} \divides k$, was $g^k \in \langle g^\frac{n}{m} \rangle$ impliziert.
	Dann ist aber $U \subseteq \langle{g^\frac{n}{m}} \rangle$ und mit $\lvert U \rvert = m = \lvert \langle g^\frac{n}{m} \rangle \rvert$ folgt $U = \langle{g^\frac{n}{m}} \rangle$.
\end{proof}

\begin{defn}
	Sei $H$ eine Untergruppe der Gruppe $G$.
	Dann sei für \(x, y \in G\) \[x \sim_{H} y :\Leftrightarrow \text{\(x = yh\) für ein \(h \in H\)}\]
	Dies ist infolge der Gruppenaxiome für \(H\) eine Äquivalenzrelation auf \(G\).
\end{defn}

\begin{defn}
	Die Äquivalenzklassen bezüglich $\sim_{H}$ heißen \emph{Linksnebenklassen}.
\end{defn}

\begin{notation}
	Für $a \in G$ sei $aH := \{ah : h \in H\}$.
\end{notation}

\begin{nb}
	Es gelten folgende Eigenschaften: 
	\begin{itemize}
	\item Die Abbildung $H \to aH, h \mapsto ah$ ist eine Bijektion mit der Umkehrung $aH \to H, b \mapsto a^{-1}b$.
		Besonders gilt: $\vert aH \vert = \vert H \vert$ für alle $a \in G$.
	\item $aH \neq bH \Rightarrow aH \cap bH = \emptyset$, d.h. sie sind disjunkt. \newline $\lceil x \in aH \neq bH \Rightarrow x = ah_{1} = bh_{2}$ für $h_{1},h_{2} \in H \Rightarrow a=bh_{2}h_{1}^{-1} \in bH \Rightarrow ah= b(h_{2}h_{1}^{-1}h) \in bH$ für alle $h \in H \Rightarrow aH \subset bH$. Ähnlich gilt $bH \subset aH$.
		Daraus folgt $aH=bH.\rfloor$ % TODO Äquivalenzklassen sind immer disjunkt. Wieso der Aufwand?
	\end{itemize}
\end{nb}

\begin{defn}
	$G/H = \lbrace aH : a \in G \rbrace$ ist die \defi{Menge der Linksnebenklassen}.
	Der \defi{Index} von $H$ ist die Mächtigkeit von $G/H$.
	Wir notieren ihn als \([G:H] := |G/H|\).
\end{defn}

\begin{nb}
	\begin{itemize}
 	\item $\vert G \vert = [G:H]\vert H\vert$
 	\item Analog ist $a \mathrel{\prescript{}{H}{\sim}} b$ mit \[a \mathrel{\prescript{}{H}{\sim}} b \Leftrightarrow \text{\(a=hb\) für ein \(h \in H\)}\] ("rechtsäquivalent bzgl. \(H\)") eine Äquivalenzrelation auf \(G\).
		Die \emph{Rechtsnebenklassen} sind $Ha=\lbrace ha : h \in H\rbrace$ mit $a \in G$ und für ist $a \in G$ \[aH \rightarrow Ha, x \mapsto a^{-1}xa\] eine Bijektion.
	\end{itemize}
\end{nb}

\begin{defn}
	$H \backslash G$ ist die \emph{Menge der Rechtsnebenklassen}. Dann gilt: $\vert H \backslash G \vert = \vert G/H \vert$
	(Bijektion: $H \backslash G \rightarrow G/H, Hb \mapsto b^{-1}H$)
\end{defn}

\begin{lemma} 
	Die Funktion $$f: H\backslash G \to G/H, \quad Hb \mapsto b^{-1}H$$ ist eine Bijektion.
\end{lemma}

\begin{proof}
	Zunächst ist \(f\) wohldefiniert, da für $b_{1},b_{2} \in G$ mit $Hb_{1}=Hb_{2}$ (das heißt \(b_1 \mathrel{\prescript{}{H}{\sim}} b_2\)) ein $h \in H$ mit $b_1 = hb_2$ exitiert.
	Somit ist $b_{1}^{-1}H = (hb_2)^{-1}H = b_{2}^{-1}hH = b_{2}^{-1}H$.

	Die Wohldefiniertheit von $$g: G/H \to H\backslash G, \quad aH \mapsto Ha^{-1}$$ folgt analog.
	Nun ist $(g \circ f)(Hb) = g(b^{-1}H)=H(b^{-1})^{-1}=Hb$ und somit $g \circ f = \id_{H \backslash G}$.
	Wiederum analog folgt $f \circ g = \id_{G \slash H}$ und somit die Bijektivität von \(f = g^{-1}\).
\end{proof}

\begin{thm}
	Es seien $H$, $K$ Untergruppen von $G$ mit $K \subset H \subset G$.
	Dann gilt: $$[G:K]=[H:K]\cdot[G:H]$$
\end{thm}

\begin{proof}
	Seien \(a_i \in i\) für \(i \in G/H\) und \(b_j \in j\) für \(j \in H/K\) Repräsentantensysteme.
	Dann ist \[G = \bigcup_{i \in G/H} a_iH = \bigcup_{i \in G/H} a_i\left(\bigcup_{j\in H/K} b_jK\right) = \bigcup_{(i, j) \in G/H\times H/K} a_ib_jK.\]
	Falls nun \(a_ib_jK = a_{i'}b_{j'}K\) gilt, so ist \(a_iH \cap a_{i'}H \supseteq a_ib_jK = a_{i'}b_{j'}K \neq \emptyset\) und somit \(a_iH = a_{i'}H\).
	Daraus folgt \(a_i = a_{i'}\) bzw. \(i = i'\).\footnote{Man könnte auch einfach \(i = a_iH = a_{i'}H = i'\) folgern, aber das würde einige eher verwirren, fürchte ich.}

	Damit wiederum erhalten wir \(b_jK = a_i^{-1}(a_ib_jK) = b_{j'}K\), also \(j = j'\).
	Folglich bilden die \(a_ib_j\) für \((i, j) \in G/H\times H/K\) ein Repräsentantensystem für \(G/K\) und \[G/H\times H/K \to G/K, \quad (i, j) = (a_iH, b_jK) \mapsto a_ib_jK\] ist eine Bijektion.\footnote{Für alle fixen Repräsentantensysteme \(a_i\) und \(b_j\). Wohldefiniertheit ohne selbige gilt nicht, wie man sich anhand von \(S_2 = G = H \supsetneq K\) leicht überlegen kann.}
	Somit haben wir \[[G:K] = |G/K| = |G/H \times H/K| = |G/H|\cdot|H/K| = [G:H]\cdot[H:K].\qedhere\]
\end{proof}

\begin{kor}[Satz von Lagrange]
	Es gilt $\vert G\vert = [G:H]\cdot\vert H\vert$ für jede Untergruppe $H$ von $G$.
	Besonders gilt für endliche Gruppen $G$: $\vert H\vert$ teilt $\vert G\vert$ und $\ord(g)$ ist ein Teiler von $\vert G\vert$ für jedes $g\in G$.
\end{kor}

\begin{proof}
	Setze $K = \{e\}$ im vorigen Satz.
\end{proof}

\begin{ex}
	Sei $G$ endlich mit $\vert G\vert=p$ Primzahl.
	Dann existiert ein $g \in G$ mit $g \neq e$, also $\ord(g)=p$.
	Somit besteht $\langle g\rangle$ aus genau $p$ Elementen.
	Also $\langle g\rangle=G$ und $G$ ist die von $g$ erzeugte zyklische Gruppe.
\end{ex}

\section{Normale Untergruppen und Gruppenhomomorphismen}

\begin{thm}
	Es sei $G$ eine Gruppe und $H\subseteq G$ eine Untergruppe.

	Die folgenden Bedingungen sind äquivalent:
	\begin{enumerate}[label=(\roman*)]
	\item Es gilt $bH=Hb$ für alle $b\in G$.
	\item Es gilt $b^{-1}Hb=H$ für alle $b\in G$.
	\item Es gilt $b^{-1}hb \in H$ für alle $b\in G$ und $h\in H$.
	\end{enumerate}
\end{thm}

\begin{defn}
	Eine Untergruppe $H$, die eine der Bedingungen (i)-(iii) erfüllt, nennt man eine normale Untergruppe (oder Normalteiler) von $G$.
\end{defn}

\begin{proof}
	(i)$\Rightarrow$(ii):
	Es sei $b \in G$ mit $bH=Hb$.
	Dann gibt es für alle $x \in b^{-1}Hb$ ein \(h \in H\) mit
	\begin{align*}
		x=b^{-1}hb &\Rightarrow bx=hb \in Hb=bH\\
		&\Rightarrow \text{Es gibt ein \(h' \in H$ mit $bx=bh'\).}\\
		&\Rightarrow x=b^{-1}bh'=h'\in H\\
		&\Rightarrow b^{-1}Hb \subseteq H
	\end{align*}
	Für $h' \in H$ ist $bh' \in bH = Hb$, woraus $bh'=hb$ für ein $h \in h$ folgt.
	Deshalb ist $h'=b^{-1}hb \in b^{-1}Hb$, wodurch $H \subseteq b^{-1}Hb$ gilt.

	(ii) $\Rightarrow$ (iii) ist klar.

	(iii) $\Rightarrow$ (i):
	Für beliebige $b \in G$, $h\in H$ gilt einerseits
	\[bh=bh(b^{-1}b)=((b^{-1})^{-1}hb^{-1})b \in Hb\]
	und somit \(bH \subseteq Hb\) und andererseits \(hb=bb^{-1}hb \in bH,\) also \(Hb \subseteq bH\).
	Daraus folgt $bH=Hb$.
\end{proof}

\begin{nb}
	Ist $N$ normal, so gilt $(aN)(bN)=abN$.
\end{nb}

\begin{proof}
	$(aN)(bN)=(Na)(bN)=N(ab)N=(abN)N=abN$
\end{proof}

\begin{ex}
	\begin{enumerate}
	\item Die trivialen Untergruppen $\{e\}$ und $G$ sind normal.
	\item Jede Untergruppe einer abelschen Gruppe ist normal.
	\item $\SL_n(K)$ ist eine normale Untergruppe von $\GL_n(K)$ für jeden Körper $K$.
	\end{enumerate}
\end{ex}

\begin{defn}
	Eine Gruppe $G$ sodass $\{e\}$ und $G$ die einzigen normalen Untergruppen sind, nennt man \defi{einfache Gruppe}.
\end{defn}

\begin{defn}
	Es seien $(G,\cdot_G)$ und $(H,\cdot_H)$ Gruppen.
	Ein Gruppenhomomorphismus von $G$ in $H$ ist eine Abbildung $f:G\rightarrow H$, sodass	$$f(a\cdot_{G}b)=f(a)\cdot_{H}f(b)$$ für alle $a,b\in G$ gilt.
\end{defn}

\begin{lemma}[Eigenschaften von Homomorphismen]
	\begin{enumerate}
	\item Das neutrale Element bleibt erhalten: $f(e_G)=e_H$
	\item Homomorphismen kommutieren mit Inversen: $f(g^{-1})=f(g)^{-1}$ für alle $g\in G$
	\item Seien $a_1: G_1\rightarrow G_2$ und $a_2: G_2\rightarrow G_3$ Gruppenhomomorphismen, dann ist $a_2\circ a_1: G_1\rightarrow G_3$ ein Gruppenhomomorphismus.
	\end{enumerate}
\end{lemma}

\begin{proof}
	\begin{enumerate}
	\item Für beliebiges \(g \in G\) gilt
		\begin{align*}
			f(e_G) &= f(e_G)\cdot_H(f(g)\cdot_Hf(g)^{-1}) = (f(e_G)\cdot_Hf(g))\cdot_Hf(g)^{-1}\\
			&= f(e_G\cdot_Gg)\cdot_Hf(g)^{-1} = f(g)\cdot_Hf(g)^{-1} = e_H.
		\end{align*}
	\item Für beliebiges \(g \in G\) gilt \[e_H=f(e_G)=f(g\cdot_G g^{-1})=f(g)\cdot_{H}f(g^{-1})\] und somit \(f(g^{-1})=f(g)^{-1}\).
	\item Für beliebige \(g, h \in G\) gilt
		\begin{align*}
			(a_2\circ a_1)(gh) &= a_2(a_1(gh)) = a_2(a_1(g)a_1(h)) = a_2(a_1(g))a_2(a_1(h))\\
			&= (a_2\circ a_1)(g)(a_2\circ a_1)(h).\qedhere
		\end{align*}
	\end{enumerate}
\end{proof}

\begin{ex}
	\begin{enumerate}
	\item Für jedes $g \in G$ ist
		$$\mathbb{Z} \rightarrow G, \quad n \mapsto g^n$$
		ein Gruppenhomomorphismus.
	\item Die Exponentialfunktion
		$$(\mathbb{R},+)\rightarrow(\mathbb{R}^*,\cdot), \quad x\mapsto e^x$$
		ist ein Gruppenhomomorphismus.
	\item Für $(G,+)$ abelsch ist
		$$G\rightarrow G, \quad a\mapsto na = a+\dots+a \text{ (\(n\)-mal)}$$
		ein Gruppenhomomorphismus.
	\end{enumerate}
\end{ex}

\begin{defn}
	Ist $N$ eine normale Untergruppe von $G$, so ist $G/N$ eine Gruppe, die man als Faktorgruppe von $G$ bezüglich $N$ bezeichnet (oder von $N$ in $G$).
	Die natürliche Projektion $$G\rightarrow G/N, \quad a\mapsto aN$$ ist dann ein surjektiver Gruppenhomomorphismus.
\end{defn}

\begin{defn}
	$G$, $H$ Gruppen, $f: G \rightarrow H$ Gruppenhomomorphismus.
	\begin{itemize}
	\item Das \defi{Bild von $f$} ist $\im(f) := \{f(g) | g \in G\} \subseteq H$.
	\item Der \defi{Kern von $f$} ist $\ker(f) := \{g \in G | f(g) = e_H\} \subseteq G$.
	\end{itemize}
\end{defn}

\begin{lemma}
	Für jeden Gruppenhomomorphismus $ f: G \rightarrow H$ gilt:
	\begin{enumerate}[label=(\roman*)]
	\item $\im(f)$ ist eine Untergruppe von $H$.
	\item $\ker(f)$ ist eine normale Untergruppe von $G$.
	\item $f$ ist surjektiv genau dann, wenn $\im(f) = H$ ist.
	\item $f$ ist injektiv genau dann, wenn $\ker(f) = \{e_G\}$ ist.\footnote{vgl. Übung 2, Aufgabe 1.}
	\end{enumerate}
\end{lemma}

\begin{proof}
	\begin{enumerate}[label=(\roman*)]
	\item $\im(f)$ ist wegen der Definition von Gruppenhomomorphismen unter $\cdot_H$ abgeschlossen, enthält $e_H = f(e_G)$ (Eigenschaft (i)) und die Inversen aller seiner Elemente (Eigenschaft (ii)).
	\item Für alle $a, b \in \ker(f)$ gilt
		$$ f(ab^{-1}) = f(a)f(b^{-1}) = f(a)(f(b))^{-1} = ee^{-1} = e.$$
		Somit ist $\ker(f)$ eine Untergruppe.
		Für jedes $ g \in \ker(f)$ und $x \in G$ gilt
		$$ f(x^{-1}gx) = f(x^{-1})f(g)f(x) = f(x^{-1}) e f(x) = f(x^{-1})f(x) = f(x^{-1}x) = f(e) = e.$$
		Also ist $x^{-1}gx \in \ker(f) $ und somit ist $\ker(f)$ ein Normalteiler.
	\item Dies ist exakt die Definition der Surjektivität.
	\item Dass aus Injektivität die Trivialität des Kernes folgt, ist klar.
		Umgekehrt gilt für alle \(a, b \in G\) mit \(f(a) = f(b)\), dass \[e_H = f(a)f(b)^{-1} =f(a)f(b^{-1}) = f(ab^{-1})\] ist.
		Somit haben wir \(ab^{-1} \in \ker(f) = \{e_G\}\), also \(a = b\) und folglich die Injektivität von \(f\). \qedhere
	\end{enumerate}	
\end{proof}

\begin{nb}
	\begin{itemize}
	\item Ein bijektiver Homomorphismus wird \emph{Isomorphismus} genannt. Die Umkehrfunktion ist dann wieder ein Isomorphismus.
	\item Ein Gruppenhomomorphismus $G \rightarrow G$ heißt \emph{Endomorphismus von G}.
	\item Ein Gruppenisomorphismus $G \rightarrow G$ heißt \emph{Automorphismus von G}.
	\end{itemize}
\end{nb}

\begin{defn}
	$G/N$ heißt \emph{Faktorgruppe von $N$ in $G$}, wenn $N$ normale Untergruppe von $G$ ist.
	Die Verknüpfung ist durch $aN \cdot bN = abN$ mit neutralem Element $eN = N$ gegeben.
	Die \emph{natürliche Projektion} $\pi : G \rightarrow G/N, a \mapsto aN$ ist ein surjektiver Gruppenhomomorphismus mit $\ker(\pi) = N, \im(\pi) = G/N$.
\end{defn}

\begin{thm}[Homomorphiesatz]
	Es sei $f: G \rightarrow G'$ ein Gruppenhomomorphismus und \(N\) ein Normalteiler von \(G\).
	Ist \(N\) im Kern von \(f\) enthalten, so gibt es genau einen Gruppenhomomorphismus $\bar{f}: G/N \rightarrow G'$, sodass das folgende Diagramm kommutiert:
	\[\begin{tikzcd}
		G \arrow[rd,"\pi"] \arrow[rr,"f=\bar f\circ \pi"] & & G'\\
		& G/N \arrow[ru,"\bar f"] &
	\end{tikzcd}\]
\end{thm}

\begin{nb}
	Es gilt $\ker(\bar{f}) = \pi(\ker(f))$ und $\im(\bar{f}) = \im(f)$. 
\end{nb}

\begin{kor}
	Wenn $f: G \rightarrow G'$ ein surjektiver Homomorphismus ist, dann ist $\bar{f}: G/\ker(f) \rightarrow G'$ ein Isomorphismus.
\end{kor}

\begin{proof}[Beweis des Satzes]
	Zunächst muss für ein solches \(\bar f\) und jedes \(aN \in G/N = \pi(G)\)
	\[\bar f(aN) = \bar f(\pi(a)) = f(a)\]
	gelten.
	Hieraus folgt bereits die Eindeutigkeit im Falle der Existenz.

	Wir zeigen noch, dass $\bar{f}$ hierdurch wohldefiniert ist.
	Es seien $a, b \in G$ mit $aN = bN$.
	Dann gilt $a \sim_{N} b$ und somit $b^{-1}a \in N \subseteq \ker(f)$, insbesondere also \(f(b^{-1}a) = e\).
	Daraus folgt $$f(a) = f(ea) = f(b(b^{-1}a)) = f(b) f(b^{-1}a) = f(b)e = f(b).$$
	Die Abbildung $\bar{f}: G/N \rightarrow G'$ existiert somit und ist offensichtlich ein Gruppenhomomorphismus, denn es ist
	$$\bar{f}((aN)(bN)) = \bar{f}(abN) = f(ab) = f(a)f(b) = \bar{f}(aN)\bar{f}(bN)$$ für alle $a, b \in G$.
\end{proof}

\begin{proof}[Beweis der Bemerkung]
	Zum Einen haben wir
	\begin{align*}
		aN \in \ker(\bar{f}) &\Leftrightarrow \bar f(aN) = e_{G'} \xLeftrightarrow{\text{Def. von \(\bar{f}\)}} f(a) = e_{G'}\\
		&\Leftrightarrow a \in \ker(f) \xLeftrightarrow{\text{Def. von \(\pi\)}} aN = \pi(a) \in \pi(\ker(f))
	\end{align*}
	und zum Anderen ist
	\[\im(f) = \im(\bar{f} \circ \pi) = \bar{f}(\im(\pi)) \stackrel{\text{\(\pi\) surj.}}{=} \bar{f}(G/N) = \im(\bar{f}).\qedhere\]
\end{proof}

\begin{thm}[1. Isomorphiesatz]
	Es seien $G$ eine Gruppe, $H \subseteq G$ eine Untergruppe und $N$ ein Normalteiler von $G$.
	Dann ist $NH = \{nh | n \in N, h \in H\}$ Untergruppe von $G$ und $N \cap H$ ein Normalteiler von $H$.
	Ferner ist $H/(N \cap H) \rightarrow (NH)/N, a(N \cap H) \mapsto aN$ ein Isomorphismus.
\end{thm}

\begin{proof}
	Es seien $n_1, n_2 \in N, h_1, h_2 \in H$. Dann gilt:
	$$ (n_1h_1)(n_2h_2)^{-1} = n_1h_1h_2^{-1}n_2^{-1} = n_1(h_1h_2^{-1}n_2^{-1}) = \dotsb $$
	[Da $N$ normal, gilt $h_1h_2^{-1}N = Nh_1h_2^{-1}$.
	Es gibt also ein $n_3 \in N$ mit $h_1h_2^{-1}n_2^{-1} = n_3h_1h_2^{-1}$.]
	$$ \dotsb = (n_1n_3)(h_1h_2^{-1}) \in NH.$$
	Daraus folgt, dass $NH$ Untergruppe von $G$ ist.
	Nun betrachten wir $f: H \rightarrow (NH)/N, a \mapsto aN = Na$.
	$f$ ist ein surjektiver Gruppenhomomorphismus mit $\ker(f) = N \cap H$.
	Aus dem Homomorphiesatz (+ Korollar) folgt dann, dass $\bar{f}: H/(N \cap H) \rightarrow (NH)/N$ ein Isomorphismus ist.
\end{proof}

\begin{thm}[2. Isomorphiesatz]
	Es seien \(M\), \(N\) normale Untergruppen einer Gruppe $G$.
	Gilt $N \subseteq M$, so ist $M/N$ eine normale Untergruppe von $G/N$ und die Abbildung $(G/N)/(M/N) \rightarrow G/M, (aN)M/N \mapsto aM$ ist ein Isomorphismus.
\end{thm}

\begin{proof}
	$f: G/N \rightarrow G/M, aN \mapsto aM$ (wohldefiniert wegen $N \subseteq M$) ist surjektiver Gruppenhomomorphismus mit $\ker(f) = M/N$.
	Die Aussage folgt dann aus dem Korollar zum Homomorphiesatz.
\end{proof}

\begin{notation}
	Seien $A, B, C$ Gruppen, $\alpha: A \rightarrow B$, $\beta: B \rightarrow C$ Gruppenhomomorphismen.
	\begin{itemize}
	\item Falls $\im(\alpha) = \ker(\beta)$, so sagt man, dass die Folge $A \stackrel{\alpha}{\longrightarrow} B \stackrel{\beta}{\longrightarrow} C$ bei $B$ \emph{exakt} ist.
	\item $\alpha$ ist surjektiv, falls $A \stackrel{\alpha}{\longrightarrow} B \longrightarrow \{e\}$ bei $B$ exakt ist. \\
		\begin{tabular}{p{4.3cm}p{.1cm}p{.3cm}l}
			& $b$ & $\mapsto$ & $e$
		\end{tabular}
	\item $\alpha$ ist injektiv, falls $\{e\} \longrightarrow A \stackrel{\alpha}{\longrightarrow} B$ bei $A$ exakt ist. \\
		\begin{tabular}{p{3.1cm}p{.1cm}p{.3cm}l}
			& $e$ & $\mapsto$ & $e_A$
		\end{tabular}
	\item Notation: $\{e\} =: 1$. Eine exakte Folge der Form
		$$ 1 \longrightarrow A \stackrel{\alpha}{\longrightarrow} B \stackrel{\beta}{\longrightarrow} C \longrightarrow 1$$
		(d.h. $\alpha$ injekiv, $\beta$ surjekiv und $\im(\alpha) = \ker(\beta)$), heißt \emph{kurze exakte Folge}.
	\end{itemize}
\end{notation}

\begin{ex}[für kurze exakte Folgen]
	\begin{itemize}
	\item $1 \longrightarrow N \stackrel{\alpha}{\longrightarrow} G \stackrel{\beta}{\longrightarrow} G/N \longrightarrow 1$ ist exakt für jeden Normalteiler $N$ von $G$.
	\item Für abelsche \(G\) ($\{e\} =: 0$) ist \(0 \longrightarrow H \stackrel{\alpha}{\longrightarrow} G \stackrel{\beta}{\longrightarrow} G/H \longrightarrow 0\) exakt für jede Untergruppe $H$ von $G$.
	\end{itemize}
\end{ex}

\section{Produkte von Gruppen}

\begin{defn}
	Es sei $(G_i)_{i \in I}$ eine Familie von Gruppen.
	Das \defi{äußere direkte Produkt} der Familie ist das kartesische Produkt $\prod_{i \in I}G_i$ mit der Verknüpfung $(a_i)_{i \in I} (b_i)_{i \in I} = (a_ib_i)_{i \in I}$.
	Das neutrale Element ist $(e_i)_{i \in I}$, wobei die $e_i \in G_i$ jeweils neutral sind.
\end{defn}

\begin{notation}
	\begin{itemize}
	\item $G_1 \times G_2 \times \dotsb \times G_n$ für endliche Produkte.
	\item $G_1 \oplus G_2 \oplus \dotsb \oplus G_n$ für endliche Produkte, $G_i$ abelsch [additiv].
	\end{itemize}
\end{notation}

\begin{lemma}
	Für jedes $i_0 \in I$ ist die Teilmenge
	$$ \overline{G_{i_0}} = \{(b_i)_{i \in I} \in \prod_{i \in I}G_i | b_i = e_i \text{ für } i \neq i_0\} $$
	ein Normalteiler von $\prod_{i \in I}G_i$, isomorph zu $G_{i_0}$.
\end{lemma}

\begin{proof}
	$\overline{G_{i_0}}$ ist der Kern des Gruppenhomomorphismus
	$$p_{i_0}: \prod_{i \in I}G_i \rightarrow \prod_{i \in I-\{i_0\}}G_i, \quad (b_i)_{i \in I} \mapsto (b_i)_{i \in I-\{i_0\}}.$$ % TODO Minus für Mengendifferenzen? Ernsthaft?
	Ferner ist
	$$j_{i_0}: G_{i_0} \rightarrow \prod_{i \in I}G_i, \quad a \mapsto (b_i)_{i \in I} \text{ mit \(b_{i_0}=a\) und \(b_i = e_i \text{ für } i \neq i_0\)}$$
	ein Homomorphismus.
	Das Bild ist $\overline{G_{i_0}}$ und $j_{i_0}$ ist injektiv, weil
	$$(b_i)_{i \in I} = (e_i)_{i \in I} \Leftrightarrow b_i = e_i \text{ für alle } i \in I,$$
	also \(j_{i_0}(a) = (e_i)_{i \in I}\) nur für \(a = e_{i_0}\).
\end{proof}

\begin{nb}
	\begin{itemize}
		\item $1 \longrightarrow G_{i_0} \stackrel{j_{i_0}}{\longrightarrow} \prod_{i \in I}G_i \stackrel{p_{i_0}}{\longrightarrow} \prod_{i \in I-\{i_0\}}G_i \longrightarrow 1$ ist kurze exakte Folge.
		\item Der Beweis liefert $\overline{G_{i_0}} \cap \langle \bigcup_{i \in I-\{i_0\}}\overline{G_i} \rangle = \{e\}$.
	\end{itemize}
\end{nb}

\begin{defn}
	Sei $G$ eine Gruppe und $(N_i)_{i\in I}$ eine Familie von normalen Untergruppen, sodass gilt:
	\begin{enumerate}[label=(\roman*)]
		\item $G=\langle\bigcup_{i\in I} N_i\rangle$
		\item $N_{i_0} \cap \langle\bigcup_{i\in I \setminus \{i_0\}} N_i\rangle = \{e\}$ für jedes $i_0\in I$.
	\end{enumerate}
	Dann ist $G$ das \emph{innere Produkt} von $(N_i)_{i\in I}$.
\end{defn}

\begin{lemma}
	Es sei $G$ das innere Produkt von $(N_i)_{i\in I}$.
	Dann gilt $ab=ba$ für $a\in N_i, b\in N_j$ mit $i\neq j$.
\end{lemma}

\begin{proof}
	Man rechnet: $$ab\inv{ba}=ab\Inv{a}\Inv{b}=(ab\Inv{a})\Inv{b}=a(b\Inv{a}\Inv{b}) \in N_i\cap N_j$$
	Somit folgt $ab\inv{ba}=e$, also $ab=ba$.
\end{proof}

Wir werden uns demnächst nur mit endlichen Produkten beschäftigen.

\begin{lemma}
	Sei $G$ Gruppe, $N_1,\dots,N_r$ normale Untergruppen von $G$.
	Dann ist $G$ genau dann das innere Produkt von $N_1,\dots,N_r$, wenn gilt:
	\begin{enumerate}[label=(\roman*)']
		\item $G=N_1\dots N_r$
		\item Die Darstellung $a=n_1\dots n_r$ mit $n_j\in N_j$ ist für jedes $a\in G$ eindeutig bestimmt.
	\end{enumerate}
\end{lemma}

\begin{proof}
	(i) äquivalent (i)': Dies folgt aus $$\langle\bigcup_{i=1}^r N_i\rangle = \{a_1^{\xi_1}\dots a_k^{\xi_k}: a_j\in N_1\cup\dots \cup N_r, \xi_j = \pm 1\}$$ und der Tatsache, dass die $N_j$ normale Untergruppen sind, die für paarweise verschiedene $j\neq j'$ kommutieren.

	(ii)' impliziert (ii): O.B.d.A. gelte $i_0=1$.
	Wir möchten zeigen, dass $$N_1\cap \langle N_2\cup\dots \cup N_r\rangle = N_1 \cap (N_2\dots N_r)= \{e\}$$ ist.
	Sei $x\in N_1\cap \langle N_2\dots N_r\rangle$.
	Dann gilt $x=n_1\in N_1$ und $x=n_2\dots n_r$ mit $n_i\in N_i$ für $i\geq 2$.
	Also $x=n_1e\dots e=en_2\dots n_r$ und somit $n_i=e$ für alle $i$.
	Folglich ist $x=e$.

	(ii) impliziert (ii)': Aus $n_1\dots n_r=n_1'\dots n_r'$ mit $n_i,n_i'\in N_i$ folgt $$\inv{n_1'}n_1=(n_2'\dots n_r')\inv{n_2\dots n_r} = \{e\}.$$
	Die letzte Gleichheit gilt nach (ii).
	Also folgt $n_i=n_i'$.
	Vertauschen der Reihenfolge liefert dies für $i\neq 1$.
\end{proof}

\begin{thm}
	Das innere Produkt $G$ von normalen Untergruppen $N_1,\dots,N_r$ ist zum äußeren Produkt $N_1\times\dots \times N_r$ kanonisch isomorph.
	Folgerung: Wir brauchen nicht zwischen den $\overline{G_i}=N_i$ und den $G_i$ zu unterscheiden.
\end{thm}

\begin{proof}
	Wir definieren $$\pi: N_1\times\dots\times N_r \longrightarrow G,\quad (n_1,\dots,n_r)\mapsto n_1\dots n_r$$
	Da das Bild von $\pi$ die Untergruppe $N_1\dots N_r \overset{(i)'}{=} G$ ist, ist $\pi$ surjektiv.
	Aus (ii)' folgt, dass $\pi$ injektiv ist.
	$\pi$ ist auch ein Gruppenhomomorphismus:
	\begin{align*}
		\pi((m_1,\dots,m_r)\cdot (n_1,\dots,n_r)) &= \pi((m_1n_1,\dots,m_rn_r)) = m_1n_1\dots m_rn_r\\
		&= m_1\dots m_rn_1\dots n_r = \pi(\bar m)\pi(\bar n)
	\end{align*}
	Dabei gilt die vorletzte Gleichheit, weil wir Elemente aus verschiedenen Untergruppen vertauschen dürfen.
\end{proof}

\begin{defn}
	Für \(N\), \(H\) Gruppen ist eine Gruppenerweiterung von $N$ durch eine kurze, exakte Folge von Gruppen
	\[ \begin{tikzcd}
		1 \arrow[r] & N \arrow[r, "i"] & G \arrow[r, "p"] & H \arrow[r] & 1
	\end{tikzcd} \]
	gegeben, wobei $i$ injektiv, $p$ surjektiv, $\im i = \ker p$ ($\nicefrac{G}{N} \cong H$). %Mit Vorsicht zu genießen. Korrekt wäre G/(im i) \cong H. Im Allgemeinen ist N keine Untergruppe von G.
\end{defn}

\begin{ex}
	Das direkte Produkt $G=N\times H$ ist immer eine, jedoch i.A. nicht die einzige, Gruppenerweiterung.
\end{ex}

\begin{defn}
	Es seien $G$ eine Gruppe, $N\subseteq G$ Normalteiler von $G$ und $H\subseteq G$ beliebige Untergruppe, sodass gilt:
	$$G=NH \quad \text{und} \quad N\cap H = \{e\}$$
	Dann ist $G$ das \emph{(innere) semidirekte Produkt} von $N$ und $H$, Notation: $G=N\rtimes H$.
\end{defn}

\begin{nb}
	Jedes $a\in G$ hat eine eindeutige Darstellung $a=nh$ mit $n\in N, h\in H$.
	Dies liefert eine Bijektion $$N\times H \to G=N\rtimes H, (n,h)\mapsto nh.$$
\end{nb}

\begin{thm}
	Für jedes $h\in H$ ist die Konjugationsabbildung $$\gamma_h: N\to N, \quad n\mapsto hn\Inv{h}$$ ein Automorphismus von $N$.
	Dies ergibt den Homomorphismus $$H\overset{\gamma}{\longrightarrow} \Aut(N), \quad h\mapsto \gamma_h$$
\end{thm}

\begin{proof}
	$\gamma$ ist Homomorphismus:
	\begin{align*}
		\gamma_{h_1h_2}(n)&=(h_1h_2)n\inv{h_1h_2}=h_1h_2n\Inv{h_2}\Inv{h_1} \\
		&= h_1\gamma_{h_2}(n)\Inv{h_1}=(\gamma_{h_1}\circ\gamma_{h_2})(n)
	\end{align*}
	für $n\in N$.
\end{proof}

\begin{nb}
	\begin{enumerate}[label=(\arabic*)]
	\item Das semidirekte Produkt von $N$ und $H$ wird von $\gamma: H\to \Aut(N)$ eindeutig (bis auf Isomorphie) festgelegt.
		Es gilt nämlich: $$(n_1h_1)(n_2h_2)=n_1(h_1n_2\Inv{h_1})h_1h_2=n_1\gamma_{h_1}(n_2)h_1h_2$$ wobei $$n_1\gamma_{h_1}(n_2)\in N, h_1h_2\in H.$$
                Damit ist $G$ zum äußeren semidirekten Produkt $N \rtimes_{\gamma_h} H$ isomorph.
	\item Umgekehrt definiert ein Homomorphismus $\gamma: H\to\Aut(N)$ immer ein (äußeres) semidirektes Produkt $N\rtimes H$:
		\begin{gather*}
			G=(N\times H, \cdot_\gamma) \\
			(n_1,h_1)\cdot_\gamma (n_2,h_2)=(n_1\gamma_{h_1}(n_2), h_1h_2)
		\end{gather*}
		ist eine Gruppe mit neutralem Element $(e_N, e_H)$ und inversen Elementen $$\Inv{(n,h)}=(\gamma_{\Inv{h}}(\Inv{n}),\Inv{h})$$
		Definiere:
		\begin{align*}
			N^*&\coloneqq \{(n, e_H): n\in N\} \cong N, \\
			H^*&\coloneqq \{(e_N, h): h\in H\} \cong H
		\end{align*}
		$N^*,H^*$ sind Untergruppen von $G$, isomorph zu $N$ bzw. $H$.
		$$\pi: G\longrightarrow H, \quad (n,h)\mapsto h$$ ist ein Homomorphismus mit $\ker \pi = N^*$, also ist $N^*$ normale Untergruppe.
		Das $N^*\cap H^* = \{e\}$ ist klar und $(n,h)=(n,e_H)\cdot_\gamma (e_N, h)$ liefert $G=N^*H^*$.
	\end{enumerate}
\end{nb}

\begin{defn}
	Eine kurze exakte Folge
	\[ \begin{tikzcd}
		1 \arrow[r] & N \arrow[r, "i"] & G \arrow[r, "p"] & H \arrow[l, bend left, dashed, "s"] \arrow[r] & 1
	\end{tikzcd} \]
	spaltet, falls ein Homomorphismus $s: H\to G$ existiert, mit $p\circ s = \text{id}_H$.
	$s$ heißt ein \defi{Schnitt} von $p$.
\end{defn}

Für $G=N\rtimes H$ Gruppenerweiterung:
\[ \begin{tikzcd}
		1 \arrow[r] & N \arrow[r,"i"] & G \arrow[r,"p"] & H \arrow[r] & 1
\end{tikzcd} \]
wobei $i: N\to G, n\mapsto (n,e_H), p:G\to H, (n,h)\mapsto h$ ist $j:H\to G, h\mapsto (e_N,h)$ Schnitt. $H$ ist Untergruppe. Man rechnet $(p\circ j)(h)=p(e_N,h)=h$

\begin{thm}
	Es sei
	\[ \begin{tikzcd}
		1 \arrow[r] & N \arrow[r,"i"] & G \arrow[r,"p"] & H \arrow[r] & 1
	\end{tikzcd} \]
	eine Gruppenerweiterung, die mit einem Schnitt $s:H\to G$ spaltet. Dann ist $G$ isomorph zum (äußeren) semidirekte Produkt von $N$ und $H$, definiert durch 
        \[\gamma: H\to \Aut(N), \quad h\mapsto \gamma_h,\]
        wobei $\gamma_h : N \to N$ gegeben ist durch
        \[ i(\gamma_h(n)) = s(h)i(n)s(h)^{-1}.  \]
        % FIXME Was ist \gamma_h? Das geht nur, falls N und H eine gemeinsame Obergruppe haben. Ist hier \(i^{-1}\circ\gamma_{s(h)}\circ i\) gemeint?
        % -- Es handelt sich hier um ein äußeres semidirektes Produkt, die gemeinsame Obergruppe ist nicht vonnöten (Fabian)
        %    -- Der Fall einer gemeinsamen Obergruppe ist jedoch der einzige, in dem wir \gamma_h definiert haben. *Das* ist mein Problem. \gamma_h kann kein semidirektes Produkt definieren, ohne dass wir \gamma_h definiert haben. (Peter)
        %       -- Siehe (2) in der Bemerkung davor. Jede Abbildung \gamma: H -> Aut(N), h -> \gamma_h:= \gamma(h) definiert ein äußeres semidirektes Produkt, also eine Gruppe auf der Trägermenge NxH mit einer speziellen Multiplikation. So verstehe ich das zumindest. Dieser Satz hier schränkt die im Falle eines Schnittes möglichen Automorphismen noch etwas ein. Da i injektiv ist, lässt sich so anscheinend ein Automorphismus definieren. Reicht das nicht aus? (Fabian)
        %       -- Dieser Satz ist auch bekannt als "Splitting-Lemma for Groups". Dieser Artikel könnte weiterhelfen http://cykenleung.blogspot.de/2012/12/a-splitting-lemma-for-groups.html (Fabian)
        %       -- Das Einzige was dem Beweis dieses Satzes aus meiner Sicht fehlt ist, der Beweis der Wohldefiniertheit von \gamma (Fabian)
\end{thm}

%\begin{proof}[$\gamma_h$ ist wohldefiniert]
%  Zunächst soll nachgewiesen werden, dass $\gamma_h \in \Aut(N)$. 
%  Da nach Voraussetzung $\im(i) = \ker(p)$ gilt, ist $\im(i)$ Normalteiler von $G$. 
%  Für alle $n \in N$ und $h \in H$ gilt somit $s(h) i(n) s(h)^{-1} \in \im(i)$. 
%  Als Monomorphismus ist $i$ insbesondere ein Isomorphismus auf $\im(i)$. 
%  Im Folgenden bezeichne $i^{-1}: \im(i) \to N$ den entsprechenden Umkehrhomomorphismus. 
%  Somit lässt sich für alle $h \in H$ eine Abbildung $\gamma_h (n) := i^{-1}( s(h)i(n)s(h)^{-1} )$ definieren.
%
%  Die so definierte Abbildung ist für alle $h \in H$ ein Gruppenhomomorphismus. Denn für $n,m \in N$ gilt
%  \begin{align*}
%    \gamma_h(nm) 
%    &= i^{-1}( s(h) i(nm) s(h)^{-1}) = i^{-1}( s(h) i(n) s(h)^{-1}\, s(h) i(m) s(h)^{-1} ) \\
%    &= i^{-1}(s(h)i(n)s(h)^{-1})\, i^{-1}(s(h)i(m)s(h)^{-1}) \\
%    &= \gamma_h(n) \gamma_h(m).
%  \end{align*}
%
%  Des Weiteren ist $\gamma_h$ injektiv. 
%  Denn für $n \in \ker(\gamma_h)$ gilt $\gamma_h(n) = e_N$ und damit $e_G = i(e_N) = s(h)i(n)s(h)^{-1}$.
%  Aus der Injektivität von $i$ folgt also $n = e_N$.
%
%  Letztlich ist $\gamma_h$ auch surjektiv.
%  Für ein $n \in N$ setze $m = i^{-1}(s(h)^{-1}i(n)s(h))$.
%  Daraus folgt die Behauptung.
%\end{proof}

\begin{proof}
	Wir setzen $\rho: N\rtimes H \to G, (n,h)\mapsto i(n)s(h)$ und zeigen, dass $\rho$ ein Homomorphismus ist: Für alle $n_1,n_2 \in N$ und $h_1,h_2 \in H$ gilt
	\begin{align*}
		\rho((n_1,h_1)\cdot (n_2,h_2)) &= \rho(n_1\gamma_{h_1}(n_2), h_1h_2) = i(n_1)i(\gamma_{h_1}(n_2))s(h_1)s(h_2)\\
		&= i(n_1)s(h_1)i(n_2)\Inv{s(h_1)}s(h_1)s(h_2) = \rho(n_1,h_1)\rho(n_2,h_2).
	\end{align*}
	Es bleibt zu zeigen, dass $\rho$ bijektiv ist.
	\begin{enumerate}
	\item Injektivität: Sei $(n,h)\in \ker(\rho)$, also \(i(n)\cdot s(h) = \rho(n, h) = e_G\).
		Daraus folgt \[e_H = p(e_G) = p(i(n)\cdot s(h)) = p(i(n))\cdot p(s(h)).\]
		Wegen der kurzen exakten Folge ist jedoch $p(i(n)) = e_H$.
		Also $e_H = p(s(h)) = h$.
		Damit ist \[e_G = i(n)\cdot s(h) = i(n)\cdot s(e_H) = i(n)\cdot e_G = i(n)\] und somit $n=e_N$, da $i$ injektiv ist.
		Daraus folgt $\ker(\rho) = \{(e_N,e_H)\}$.
		Somit ist $\rho$ injektiv.
	\item Surjektivität: Sei $a\in G$ beliebig.
		Wir setzen $b := a\cdot\inv{s(p(a)}$.
		Wegen der Schnitteigenschaft von $s$ ist $p\circ s = \id_H$ und somit $p(b) = p(a)\cdot (p\circ s)(\Inv{p(a)})=p(a)\cdot\Inv{p(a)}=e_G$.
		Daraus folgt $b\in\ker(p)= \im(i)$ und somit gibt es ein $n\in N$ mit $i(n) = b$.
		Dies liefert $\rho(n,p(a)) = b\cdot s(p(a))=a\cdot s(p(a))^{-1}s(p(a))=a$. \qedhere
	\end{enumerate}
\end{proof}

\begin{ex}
\label{ex:semidirekt}

\begin{enumerate}[label=(\alph*)]
	\item Für einen beliebigen Körper $K$ zerfällt (bzw. spaltet)
		\[ \begin{tikzcd}
			 1 \arrow[r] & \SL_n(K) \arrow[r] & \GL_n(K) \arrow[r,"\det"] & K^* \arrow[r] & 1.
		\end{tikzcd} \]
		Das heißt $GL_n(K) = SL_n(K)\rtimes K^*$.
	\item Es sei $N$ eine beliebige abelsche Gruppe und $H := (\{\pm 1\},\cdot)\left[\cong(\mathbb{Z}/2\mathbb{Z},+)\right]$. Für $h\in H$ sei $\gamma_h$ definiert als $\gamma_h:N\to N, \gamma_h(n) := n^h$.

		Dann heißt $D_N := N \rtimes H$ verallgemeinerte Diedergruppe. Diese hat die Verknüpfung
		$(n_1,\epsilon_1)\cdot(n_2,\epsilon_2) = (n_1 n_2^{\epsilon_1},\epsilon_1\epsilon_2)$.
		Es ist $N \cong N\times\{1\}$ eine normale Untergruppe von $D_N$ mit Index $[D_N:N] = 2$.

	\item Spezialfall: $N = \mathbb{Z}/k\mathbb{Z}$, $D_{2k} := D_{\mathbb{Z}/k\mathbb{Z}}$ heißt Diedergruppe der Ordnung $2k$.
		Dann ist $D_{2k}$ die Symmetriegruppe des regelmäßigen \(k\)-Gons.
		Die Untergruppe $N\subset D_{2k}$ ist gerade die Menge der Drehungen.

		Ist $G$ eine von zwei Elementen $n,h\in G$ erzeugte Gruppe, also $G = \langle n, h\rangle$, mit $\ord(n) = k$ ($\cong$ Drehung) und $\ord(h) = 2$ ($\cong$ Spiegelung) und außerdem $h^{-1}nh=n^{-1}$, so gilt $G\cong D_{2k}$.
\end{enumerate}

\end{ex}

\section{Die symmetrische Gruppe}

\begin{defn}
	Für eine Menge $M\neq\emptyset$ bildet $S_M$ als Menge aller Bijektionen von $M\to M$ mit der Verkettung von Funktionen als Verknüpfung und der Identität als neutralem Element eine Gruppe, die \emph{symmetrische Gruppe von $M$}.
	Die Elemente von $S_M$ heißen \emph{Permutationen}.
\end{defn}

\begin{thm}[Satz von Cayley]
	Jede Gruppe $G$ ist isomorph zu einer Untergruppe von $S_G$.
\end{thm}

\begin{proof}
	Für jedes Element $a\in G$ ist die Linkstranslation um $a$, also $\lambda_a:G\to G, b\mapsto a\cdot b$ eine Permutation.

	Die Abbildung $\lambda:G\to S_G, a\mapsto \lambda_a$ ist ein Homomorphismus, denn für $a,b,h\in G$ gilt
	\[\lambda_{ab}(h) = abh = a(bh) = a\lambda_b(h) = \lambda_a(\lambda_b(h)) = (\lambda_a\circ\lambda_b)(h).\]

	Es sei $a\in \ker(\lambda)$.
	Dann gilt $\lambda_a = \text{id}_G$, also $ah = h$ für alle $h\in G $ und damit \[a = a\cdot e = \lambda_a(e) = e.\]
	Folglich ist der Kern trivial (\(\ker(\lambda) = \{e\}\)).
	Also ist $\lambda$ injektiv und somit $G\cong\im(\lambda)$.
\end{proof}

\begin{defn}
	Es sei $M\neq\emptyset$ eine Menge und $\sigma$ eine Permutation von $M$.
	\begin{enumerate}
	\item Der \defi{Träger} von $\sigma$ ist $\supp(\sigma) := \{m\in M: \sigma(m)\neq m\}$.
	\item Ein \defi{Zyklus} der Länge $l$ ist eine Permutation $\sigma\in S_M$ mit $l = |\supp(\sigma)|$, sodass
		\[\supp(\sigma) = \{m_1,m_2,\dots,m_l\}\text{ mit }\sigma(m_j) = m_{j+1},\; 1\leq j<l,\; \sigma(m_l) = m_1.\]
		Für einen solchen Zyklus schreiben wir $\sigma = (m_1m_2\dots m_l)$.
	\item 2-Zyklen werden \defi{Transpositionen} genannt.
	\end{enumerate}
\end{defn}


\begin{nb}
	Sind $\sigma,\tau\in S_M$ mit $\supp(\sigma)\cap \supp(\tau) = \emptyset$, so gilt $\sigma\circ\tau = \tau\circ\sigma$.
\end{nb}


\begin{defn}
	Im Spezialfall $M=\{1,2,\dots,n\}$ schreibt man $S_n := S_M$.
	Ein Element $\sigma\in S_n$ schreibt man als
	\[\sigma = \begin{pmatrix}
	 1 & 2 & \dots & n \\
	 \sigma(1) & \sigma(2) & \dots & \sigma(n) \end{pmatrix}.\]
\end{defn}

\begin{ex}
	$\sigma = \begin{pmatrix}
	1 & 2 & 3 & 4 \\
	4 & 2 & 1 & 3 \end{pmatrix} \in S_4$.
	$\sigma$ ist der 3-Zyklus $(143) = (431) = (314)$.
\end{ex}

\begin{thm}
\begin{enumerate}[label=(\alph*)]
	\item Jede Permutation $\sigma\in S_n, \sigma\neq \id$, kann als Produkt von Zyklen mit disjunkten Trägern geschrieben werden.
		Diese Darstellung ist bis auf die Reihenfolge der Faktoren eindeutig bestimmt.
	\item Für einen beliebigen l-Zyklus $(m_1m_2\dots m_l)\in S_n$ und $\sigma\in S_n$ gilt
		\[\sigma\circ(m_1m_2\dots m_l)\circ\sigma^{-1} = (\sigma(m_1)\sigma(m_2)\dots\sigma(m_l)).\]
	\item Die symmetrische Gruppe $S_n$ wird von den Transpositionen erzeugt.
\end{enumerate}
\end{thm}

\begin{proof}
	\begin{enumerate}[label=(\alph*)]
	\item Es sei $\sigma\in S_n, \sigma\neq\text{id}$. Für $m_1,m_2\in\{1,\dots,n\}=:M$ definieren wir 
		\[m_1\sim m_2 :\Leftrightarrow \text{\(m_1 = \sigma^l(m_2)\) für ein \(l\in\mathbb{Z}\)}.\]
		Dann ist $\sim$ eine Äquivalenzrelation auf $M$.
		Es sei $M = K_1\cup K_2\cup\dots\cup K_r$ die Zerlegung von $M$ in disjunkte Äquivalenzklassen.
		Wir betrachten $a_j\in K_j$ beliebig.
		Es sei $l_j$ die kleinste positive Zahl mit $\sigma^{l_j}(a_j) = a_j$.\footnote{Ein solches $l_j$ existiert aufgrund der vorausgesetzten Endlichkeit der symmetrischen Gruppe nach dem Satz von Lagrange. Es gilt also $l_j \leq \ord(\sigma)$.} %FIXME Auf den zweiten Blick glaube ich , dass sogar l_j = |K_j| gelten muss, nach Definition der Äquivalenzrelation (Fabian)
		Das liefert uns \[a_j,\sigma(a_j),\sigma^2(a_j),\dots,\sigma^{l_j-1}(a_j)\in K_j\] und diese Elemente sind paarweise verschieden. 
                %FIXME Folgendes muss noch hinzugefügt werden: 
                % BEGIN
                % Aufgrund der Konstruktion der Äquivalenzrelation gilt sogar \[ \{ a_j,\sigma(a_j),\sigma^2(a_j),\dots,\sigma^{l_j-1}(a_j) \} =  K_j\] 
                % END
                % ansonsten ist nicht klar, wieso die unten folgende Gleichheit \sigma = (a_1\dots ... gelten sollte.
                % (Fabian)
		Aus der Definition folgt, dass $\sigma\upharpoonright_{K_j}=(a_j\sigma(a_j)\dots\sigma^{l_j-1}(a_j))\in S_{K_j}$ ein Zyklus der Länge $l_j$ ist.
		Wenn wir dies für alle $j=1,\dots,r$ tun, erhalten wir:
		\[\sigma = (a_1\dots\sigma^{l_1-1}(a_1))(a_2\dots\sigma^{l_2-1}(a_2))\dots(a_r\dots\sigma^{l_r-1}(a_r)).\]
		Dies ist genau die gesuchte Darstellung, wenn man zudem $(a_j):=\text{id}$ für $l_j = 1$ definiert.

		Eindeutigkeit: Für $s\in\mathbb{N}$ sei $\sigma = \tau_1\tau_2\dots\tau_s$ eine andere Darstellung.
		Dann ist $\sigma\upharpoonright_{\supp(\tau_k)} = \tau_k$ für alle $k$ ein Zyklus.
		Damit ist $\supp(\tau_k) = K_j$ für ein $j=1,\dots,r$ und somit ist $\tau_k$ gerade ein solcher Zyklus wie in unserer Konstruktion.
		Diese ist also eindeutig.
	\item Es sei $\sigma\in S_n$ und $(m_1\dots m_k)$ ein Zyklus.
		Dann gilt für $\alpha = 1,\dots,k$
		\[(\sigma(m_1 \dots m_k)\Inv\sigma)(\sigma(m_{\alpha}))=(\sigma(m_1 \dots m_k))(m_{\alpha}) = \sigma(m_{\alpha+1})\]
		mit $m_{k+1}:=m_1$.
		Für $m\notin \{m_1, \dots, m_k\}$ gilt
		\[(\sigma(m_1 \dots m_k) \sigma^{-1}) (\sigma(m)) = ( \sigma(m_1 \dots m_k) )(m) = \sigma(m).\]
		Daraus folgt $\sigma\circ(m_1m_2\dots m_k)\circ\sigma^{-1} = (\sigma(m_1)\sigma(m_2)\dots\sigma(m_k))$.
	\item Wegen (a) genügt es zu zeigen, dass für alle $k\in\mathbb{N}$ alle $k$-Zyklen Produkte von Transpositionen sind.
		Für einen beliebigen Zyklus gilt
		\[(m_1\dots m_k) = (m_1m_2)(m_2m_3)\dots (m_{k-1}m_k).\qedhere\]
	\end{enumerate}
\end{proof}

\begin{nb}
	Es reichen deutlich weniger Transpositionen aus, um $S_n$ zu erzeugen.
\end{nb}

\begin{ex}
	\begin{itemize}
	\item $(1234) = (12)(23)(34)$. Graphisch veranschaulicht:
		\[\begin{pmatrix}
		1 & 2 & 3 & 4 \\ % (34)
		1 & 2 & 4 & 3 \\ % (23)
		1 & 3 & 4 & 2 \\ % (12)
		2 & 3 & 4 & 1
		\end{pmatrix}.\]
	\item Für $n \geq 2$ gilt bereits $S_n = \langle (1 2) , (1 \dots n) \rangle$.\footnote{Siehe Übung 3, Aufgabe G3.}
	\end{itemize}
\end{ex}

\begin{defn}
	Für $\sigma\in S_n$ ist das Signum von $\sigma$ definiert als
	\[\sgn(\sigma) := \prod_{i<j}\frac{\sigma(i)-\sigma(j)}{i-j}\qquad(\in\{\pm 1\}).\]
	Permutationen $\sigma\in S_n$ mit $\sgn(\sigma) = 1$ (bzw. $-1$) heißen gerade (bzw. ungerade) Permutationen.
\end{defn}

\begin{lemma}
	Die Abbildung $\sgn:S_n\to\{\pm 1\}$ ist ein Gruppenhomomorphismus.
\end{lemma}

\begin{proof}
	Es seien $\sigma,\tau\in S_n$ beliebig. Dann gilt 
	\begin{align*}
		\sgn(\sigma\tau) &= \prod_{i<j}\frac{\sigma\tau(i)-\sigma\tau(j)}{i-j} = \prod_{i<j}\frac{\sigma\tau(i)-\sigma\tau(j)}{\tau(i)-\tau(j)}\cdot\prod_{i<j}\frac{\tau(i)-\tau(j)}{i-j} \\
		&= \left(\prod_{\substack{i<j,\\\tau(i)<\tau(j),\\ i':=\tau(i),j':=\tau(j)}}\frac{\sigma\tau(i)-\sigma\tau(j)}{\tau(i)-\tau(j)}\cdot \prod_{\substack{i<j,\\\tau(j)<\tau(i),\\ i':=\tau(j),j':=\tau(i)}}\frac{\sigma\tau(i)-\sigma\tau(j)}{\tau(i)-\tau(j)}\right)\cdot \sgn(\tau) \\
		&=\prod_{i'<j'}\frac{\sigma(i')-\sigma(j')}{i'-j'}\cdot \sgn(\tau) \\
		&= \sgn(\sigma)\cdot \sgn(\tau).\qedhere
	\end{align*}
\end{proof}

% VL 26.5.
\begin{kor}
	Jede gerade Permutation ist das Produkt einer geraden Anzahl von Transpositionen.
	Jede ungerade Permutation ist Produkt einer ungeraden Anzahl von Transpositionen.
\end{kor}

\begin{ex}
	\(m\)-Zyklus $(n_1n_2 \dots n_m)$ ist das Produkt von $m-1$ Transpositionen:
	\[\sgn((n_1n_2 \dots n_m))=(-1)^{m-1}\]
\end{ex}

\begin{kor}
	Für $\sigma = (m_1 \dots m_{k_1})(m_{k_1+1} \dots m_{k_2}) \dots (m_{k_{r-1}+1} \dots m_{k_r})$ ist $\sgn(\sigma)=(-1)^{k_r-r}$.
\end{kor}

\begin{defn}
	Die normale Untergruppe $A_n:=\ker(\sgn)$ der geraden Permutationen wird die alternierende Gruppe genannt.
\end{defn}
	
\begin{nb}
	Index von $A_n$ in $S_n$ ist \(2\), also \(\ord(A_n)=\frac{1}{2}\ord(S_n)\), wobei $\ord(S_n)=n\cdot(n-1)\cdot\ldots\cdot1=n!$ ist.
\end{nb}
	
\begin{lemma}
	$A_n$ wird durch die 3-Zyklen von $S_n$ erzeugt.
\end{lemma}
	
\begin{proof}
	Es sei $\sigma \in A_n$ eine gerade Permutation.
	Wir schreiben $\sigma$ als Produkt von Transpositionen $\sigma =(a_1b_1)(a_2b_2)\dots(a_{2m}b_{2m})$ und betrachten \((a_{2j-1}b_{2j-1})(a_{2j}b_{2j})\) für \(1 \leq j \leq m\).
	Im Falle disjunkter Träger ist dies mit
	\begin{align*}
		(a_{2j-1}b_{2j-1})(a_{2j}b_{2j}) &= (a_{2j-1}a_{2j})(a_{2j}b_{2j-1})(a_{2j-1}a_{2j})(a_{2j}b_{2j})\\
		&= (a_{2j-1}a_{2j}b_{2j-1})(a_{2j-1}a_{2j}b_{2j})
	\end{align*}
	ein Produkt von 3-Zyklen.
	Wenn es ein gemeinsames Element gibt, also \(|\{a_{2j-1},b_{2j-1}\} \cap \{a_{2j},b_{2j}\}| = 1\) ist, kann mit \((ab) = (ba)\) o.B.d.A. $b_{2j-1}=a_{2j}$ angenommen werden und wir haben über $(a_{2_{j-1}}a_{2_j})(a_{2_j}b_{2_j})=(a_{2_{j-1}}a_{2j}b_{2j})$ einen 3-Zyklus.
	Für \(\{a_{2j-1},b_{2j-1}\} = \{a_{2j},b_{2j}\}\) ist schließlich \((a_{2j-1}b_{2j-1})(a_{2j}b_{2j}) = \id\).

	Somit kann \((a_{2j-1}b_{2j-1})(a_{2j}b_{2j})\) als Produkt von 3-Zyklen geschrieben werden und auch \(\sigma = \prod_{j = 1}^m (a_{2j-1}b_{2j-1})(a_{2j}b_{2j})\) ist ein Produkt von 3-Zyklen.
\end{proof}
	
\begin{ex}
	\begin{enumerate}
	\item $S_1 = \{(1)\}$
	\item $S_2 = \{\id, (12)\} \cong \Z/2\Z$ ist zyklisch mit Ordnung \(2\).
	\item $S_3 = \{\id, (12), (13), (23), (123), (132)\} \supsetneq A_3=\{\id, (123), (132)\} \cong \Z/3\Z$
	\item $S_3 \cong A_3\rtimes \mathbb{Z}/2\mathbb{Z} \cong \mathbb{Z}/3\mathbb{Z} \rtimes \mathbb{Z}/2\mathbb{Z} \cong D_6$ ist die Diedergruppe mit 6 Elementen.
	\end{enumerate}
\end{ex}
	 
\section{Gruppenoperationen}

\begin{defn}
	Es sei $(G, \cdot)$ eine Gruppe und $X$ eine Menge.
	Eine Operation von \(G\) auf \(X\) ist eine Abbildung
	$$G \times X \rightarrow X, \quad (g,x) \mapsto g\bullet x,$$
	welche folgende Bedingungen erfüllt:
	\begin{enumerate}
	\item $e \bullet x = x$ ($e\in G$ neutral)
	\item \((a\cdot b)\bullet x = a\bullet (b\bullet x)\) für alle \(a,b\in G\) und \(x\in X\)
	\end{enumerate}
\end{defn}

\begin{ex}
	Für einen $K$-Vektorraum \(V\) operiert die allgemeine lineare Gruppe $\GL(V)$ auf $V$:
	$$\GL(V) \times V \to V, \quad (A,v) \mapsto Av$$
\end{ex}

\begin{notation}
	Wenn $G$ auf $X$ operiert, sagt man, dass $X$ eine $G$-Menge ist.
	Bezeichnung: $G\circlearrowright X$
\end{notation}

\begin{nb}
	Für jedes Element $g\in G$ ist $$\tau_g: X \rightarrow X, \quad x \mapsto gx$$ eine Bijektion mit Inverser $\tau_{g^{-1}}$.
	Aus 1. und 2. folgt:
	$$\tau: G \rightarrow S_X, \quad g \mapsto \tau_g$$ ist ein Gruppenhomomorphismus.
	Umgekehrt definiert jeder Gruppenhomomorphismus $\tau: G \rightarrow S_X$ durch $g \bullet x := (\tau(g))(x)$ eine Gruppenoperation.
\end{nb}
	
\begin{defn} Es sei $\bullet: G \times X \to X$ eine Gruppenoperation.
	\begin{enumerate}
	\item Für $x\in X$ heißt $Gx := \{gx : g \in G\}$ die Bahn (oder Orbit) von X. 
	\item Die Untergruppe $G_x:= \{g\in G : gx=x\}$ heißt Isotropiegruppe (oder Stabilisator) von $x$.
	\item Menge der Fixpunkte unter $G$: $X^G := \{x \in X : \forall g \in G: gx=x\}$
	\item Die Gruppenoperation ist treu, falls $\tau:G \rightarrow S_X$ injektiv ist.
	\item Die Gruppenoperation ist transitiv, wenn $X$ eine Bahn ist.
	\end{enumerate}
\end{defn}
	
\begin{ex}
	\begin{enumerate}
	\item Der Satz von Cayley ergibt einen injektiven Homomorphismus:
		$$G \to S_G, \quad a \mapsto \lambda_a,$$ wobei $$\lambda_a: G \to G, \quad b \mapsto ab$$ ist.
		Das heißt $$G\times G \to G, \quad (a,b) \mapsto ab$$ ist eine treue Operation von \(G\) auf \(G\).
		Diese Operation ist transitiv, da wir $Ge=G$ haben, also $G$ eine Bahn ist.
	\item Die Rechtstranslation
		$$\varphi : G \to G, \quad b \mapsto ba^{-1}$$
		definiert für beliebiges $a\in G$ eine treue und transitive Gruppenoperation von $G$ auf $G$:
		$$G \times G \to G, \quad a \bullet b := ba^{-1}$$
	\item $G$ operiert auf sich selbst durch Konjugation.
		Für \(a \in G\) ist die Konjugation mit $a$
		$$i_a : G \to G, \quad x \mapsto axa^{-1}$$
		für alle $x \in G$ ein Automorphismus von G. Somit ist die Abbildung
		$$i: G \to \Aut(G) \subseteq S_G, \quad a \mapsto i_a$$
		ein wohldefinierter Gruppenhomomorphismus, d.h. eine Operation von $G$ auf sich selbst.
		\begin{proof}
			Für alle \(a, b, x \in G\) ist \[i_{ab}(x)=abx(ab)^{-1}=abxb^{-1}a^{-1}=a(i_b(x))a^{-1}=i_a(i_b(x))=(i_a\circ i_b)(x).\qedhere\]
		\end{proof}
		\begin{defn}
			Die Bahnen von $i$ heißen \defi{Konjugationsklassen}.
			Der Kern von $i$ heißt \defi{Zentrum} von $G$.
			\[Z(G) := \ker(i) = \{a \in G : \forall x \in G: axa^{-1}=x\} = \{a \in G : \forall x \in G: ax =xa\}\]
			Außerdem ist \[G^G = \{y \in G : \forall b \in G: byb^{-1}=y\} = \{y \in G: \forall b \in G: by = yb\} = Z(G),\]	
			d.h. $Z(G)$ ist auch die Menge der Fixpunkte.
			Der Stabilisator von $x \in G$ ist der \defi{Zentralisator} von \(x\) \[C_G(x) := G_x = \{a \in G : axa^{-1}=x\} = \{ a \in G : ax = xa\}.\]
		\end{defn}		
	\end{enumerate}
\end{ex}

\begin{defn}
	Seien $X$ und $Y$ $G$-Mengen.
	Dann ist $f: X \rightarrow Y$ ein $G$-Homomorphismus, falls $f(gx)=g(f(x))$ ist, für alle $x \in X und g \in G$.
\end{defn}

\begin{thm}
	$G \times X \rightarrow X$ Gruppenoperation. Dann ist die Abbildung
	$$p: G/G_x \to Gx, \quad gG_x \mapsto gx$$
	ein bijektiver $G$-Homomorphismus für alle $x \in X$.
\end{thm}

\begin{kor}
	Die Kardinalität der Bahn ist der Index des Stabilistors: $|Gx|=|G/G_x|=[G:G_x]$.
\end{kor}
	
\begin{proof}
	Da für beliebige \(a,b,x \in G\)
	$$aG_x=bG_x \Leftrightarrow b^{-1}a \in G_x \Leftrightarrow b^{-1}ax=x \Leftrightarrow ax=bx \Leftrightarrow p(a)=p(b)$$
	gilt, ist $p$ sowohl wohldefiniert als auch bijektiv.
	Da außerdem für beliebige \(a,b,x \in G\)
	$$p(a(bG_x))=p(ab(G_x))=abx=ap(bG_x)$$
	gilt, ist $p$ ein $G$-Homomorphismus.
\end{proof}

\begin{nb}
	Die Bahnen einer Gruppenoperation sind die Äquivalenzklassen bezüglich der Äquivalenzrelation
	\[x \sim y :\Leftrightarrow \text{\(x=gy\) für ein \(g\in G\).}\]
	D.h. die Bahnen bilden eine Partition von $X$.
	Für ein Repräsentantensystem $\{x_i\}_{i\in I}$ der Bahnen von \(X\) gilt somit die Bahnengleichung
	$$ |X| =\sum\limits_{\begin{subarray}{1}i \in I\end{subarray}}|Gx_i| =\sum\limits_{\begin{subarray}{1}i \in I\end{subarray}}[G:G_{x_i}] =|X^G|+\sum\limits_{\begin{subarray}{c}i \in I\\ x_i \notin X^G\end{subarray}}[G:G_{x_i}]. $$
\end{nb}

%VL 2.6.15
Für die Operation von \(G\) auf sich selbst durch Konjugation gilt:
\begin{thm} [Klassengleichung]
	Ist $\{x_{i}\}_{i \in I}$ ein Repräsentantensystem für die Konjugationsklassen der Gruppe \(G\), so gilt\footnote{Erinnerung: \(Z(G)=\{a \in G : \forall g \in G: ag=ga\}\) ist das Zentrum von \(G\) und \(C_G(x) = \{a \in G :ax = xa\}\) der Zentralisator von \(x\).}
	\[|G|= |Z(G)|+\sum_{x_{i}\notin Z(G)}[G:C_{G}(x_{i})].\]
        Für endliche Gruppen ist also $[G:G_{x_{i}}]=\dfrac{|G|}{|G_{x_{i}}|}$. % TODO WTF hat das mit dem Rest hier zu tun? Für alle Untergruppen gilt [G:H] = |G|/|H|. -- Beachte das Wörtchen *endlich*. I.A. gilt nur |G| = [G:H] |H|. (Fabian)
\end{thm}

\begin{nb}
	Anwendung: Existenz von Fixpunkten
\end{nb}

\begin{defn}
	Eine endliche Gruppe \(G\) wird \defi{\(p\)-Gruppe} genannt, wobei \(p\) eine Primzahl ist, wenn die Ordnung von \(G\) eine Potenz $p^n$ von \(p\) ist $(n\geq 1)$.
\end{defn}

\begin{thm}
	Das Zentrum einer \(p\)-Gruppe \(G\) enthält nicht triviale Elemente $a \neq e$.
\end{thm}

\begin{proof}
	\begin{enumerate}
	\item \(n=1\): \(\ord(G)=p\) ist eine Primzahl.
		Also ist \(G\) eine zyklische Gruppe.
		Da \(G\) abelsch ist, gilt \(Z(G) = G \neq \{e\}\).
	\item \(n\geq 2\): Aus dem Satz von Lagrange folgt, dass die Ordnung jeder Untergruppe $H$ von $G$ ein Teiler von $p^n$ ist, d.h. eine Primpotenz $p^k$ mit $0\leq k\leq n$.
		Es seien $x_{1}, \dots ,x_{m}$ Repräsentanten der Konjugationsklassen von $G$.
		Da $C_{G}(x_{i})$ eine Untergruppe ist, gilt \(|C_{G}(x_{i})|=p^{k_i}\) mit $0\leq k_{i} \leq n$ und $k_{i}=n$ nur für $x_i \in Z(G)$.

		Über die Klassengleichung ist \[p^n=|X|=|Z(G)|+\sum_{x_{i}\notin Z(G)}p^{n-k_{i}}\] mit $n-k_{i}>0$.
		Also ist $|Z(G)|= p^n-\sum_{x_{i}\notin Z(G)}p^{n-k_{i}}$ durch \(p\) teilbar und wegen $|Z(G)|\geq 1$ gilt $|Z(G)|\geq p$.
		Es gibt also mindestens $p-1 > 0$ Elemente in $Z(G)$, die vom neutralen Element verschieden sind.\qedhere
\end{enumerate}
\end{proof}

\begin{thm}
	Es sei G eine $p$-Gruppe, die auf einer endlichen Menge $X$ operiert. Dann ist
	\begin{equation*}
		|X^G|\equiv |X| \pmod{p}.
	\end{equation*}
\end{thm}

\begin{proof}
	Es sei $\{x_{1}, \dots ,x_{m}\}$ ein Repräsentantensystem der Bahnen.
	Wieder ist $|C_{G}(x_{i})|=p^{k_i}$ mit $0\leq k_{i} \leq n$ und $k_{i}=n$ nur für $x_{i} \in X^G$.
	Die Bahnengleichung liefert, dass $|X|-|X^G|=\sum_{x_{i}\notin Z(G)}p^{n-k_{i}}$ ein Vielfaches von \(p\) ist.
\end{proof}

\section{Strukturtheorie der Gruppen}

\subsection{Die sylowschen Sätze}

Der Satz von Lagrange gibt uns die Teilbarkeit $\ord(H) \divides \ord(G)$ für jede Untergruppe $H$ einer endlichen Gruppe $G$.
Aber: Wenn \(h\) ein Teiler von $\ord(G)$ ist, existiert dann auch eine Untergruppe \(H \subseteq G\) mit $\ord(H)=h$?

\begin{thm}
	Es seien $p$ eine Primzahl und $G$ eine abelsche Gruppe der Ordnung $pn$ für ein $n\in\mathbb{N}$.
	Dann existiert immer ein $a \in G$ der Ordnung $p$.
\end{thm}

\begin{proof}
	Induktion nach $n=\dfrac{|G|}{p}$
	\begin{itemize}
	\item Induktionsstart: Sei \(n=1\), d.h. $|G|=p$.
		Daraus folgt: $G=\langle a\rangle$ ist zyklisch und der Erzeuger \(a\) hat Ordnung \(p\).
	\item Induktionsschritt: Wir nehmen an, dass die Aussage für jede Gruppe der Ordnung $pm'$ mit $m'<n$ gilt.
		Es sei $h\neq e$ ein beliebiges Element von \(G\) und $m:=\ord(h)$.

		Wenn \(m\) durch \(p\) teilbar ist, enthält die zyklische Untergruppe $\langle h\rangle$ ein Element der Ordnung \(p\).

		Wenn \(p\) kein Teiler von \(m\) ist, so ist, mit \(pn = |G| = |G/\langle h\rangle||\langle h\rangle| = |G/\langle h\rangle|m\), \(p\) ein Teiler von $|G/\langle h\rangle|<pn$.
		Nach Induktionsvoraussetzung enthält die Gruppe $|G/\langle h\rangle|$\footnote{\(\langle h\rangle\) ist eine normale Untergruppe, da \(G\) abelsch ist.} ein Element $a\langle h\rangle$ der Ordnung \(p\).
		Es sei $k:=\ord(a)$ die Ordnung des gewählten Repräsentanten $a \in G$.
		Nun liefert \[(a\langle h\rangle)^k=a^k\langle h\rangle=\langle h\rangle\] die Teilbarkeit \(p = \ord(a\langle h\rangle) \divides k\).
		Daraus folgt, dass $a^{\frac{k}{p}}$ ein Element von $\langle a\rangle \subseteq G$ der Ordnung \(p\) ist.\qedhere
	\end{itemize}
\end{proof}

\begin{thm}[Cauchy]
	Ist \(p\) eine Primzahl, die die Ordnung der endlichen Gruppe \(G\) teilt, so enthält \(G\) ein Element der Ordnung \(p\).
\end{thm}

\begin{proof}
	Induktion über $\ord(G)$
	\begin{itemize}
	\item Induktionsstart: $\ord(G)=1$ ist klar
	\item Induktionsschritt: Wir nehmen an, dass die Aussage für alle Gruppen \(T\) gilt mit $\ord(T)<\ord(G)$.
		Wenn \(G\) eine Untergruppe \(H\) enthält, deren Ordnung durch \(p\) teilbar ist, so enthält \(H\) ein Element der Ordnung \(p\).
		Wir können annehmen, dass die Ordnung jeder Untergruppe $H \subsetneq G$ nicht durch \(p\) teilbar ist.
		Insbesondere gilt, im Spezialfall der Zentralisation, $p \nmid \ord(C_{G}(a))$ für alle $a \notin Z(G)$.
		Nun seien \(x_i\) für \(i \in I\) ein beliebiges Repräsentantensystem der Konjugationsklassen von \(G\).
		Die Klassengleichung gibt uns dann \[|Z(G)| = |G| - \sum_{x_i \notin Z(G)}\dfrac{|G|}{|C_{G}(x_i)|}\] und somit $p \divides \ord(Z(G))$.
		Aus der Annahme folgt dann \(Z(G)=G\).
		\(G\) ist also abelsch und die Aussage folgt aus dem vorherigen Satz.\qedhere
\end{itemize}
\end{proof}

\begin{kor}
	Die Ordnung einer endlichen Gruppe \(G\) ist genau dann eine Primpotenz für eine Primzahl \(p\), wenn die Ordnung jedes Elements von \(G\) eine Primpotenz bezüglich \(p\) ist.
\end{kor}

\begin{proof}
	\begin{itemize}
	\item["$\Rightarrow$":] Nach Lagrange gilt für jedes \(g\in G\) die Teilbarkeit \(\ord(g) \divides \ord(G)=p^n\), woraus folgt, dass $\ord(g)=p^k$ für ein \(k \leq n\) ist.
	\item["$\Leftarrow$":] Zwecks Kontraposition sei die Ordnung von \(G\) von der Form $\ord(G)=p^\alpha q$ mit $q\neq 1$ teilerfremd zu \(p\).
		Dann existiert nach dem Satz von Cauchy für jeden Primfaktor $p'$ von \(q\) ein Element $g \in G$ mit \(\ord(g)=p'\). Somit ist die Ordnung von $g$ nicht von der Form $p^k$ im Widerspruch zur Voraussetzung.
	\end{itemize}
\end{proof}

\begin{nb}
	(Unendliche) Gruppen \(G\) werden \defi{\(p\)-Gruppen}, für eine feste Primzahl \(p\), genannt, wenn für alle \(a \in G\) \[\ord(a)=p^{n(a)}\] mit einer geeigneten Funktion \(n: G \to \N\) ist.
\end{nb}

\begin{defn}
	Es sei \(p\) eine Primzahl und \(G\) eine endliche Gruppe.
	Man schreibt $|G|=p^n q$ mit $n \in \mathbb{N}$ und \(q\) teilerfremd zu \(p\).
	Eine Untergruppe \(S\) von \(G\) ist eine \defi{\(p\)-Sylowuntergruppe}, wenn $|S|=p^n$ ist.
\end{defn}

\begin{thm}[Sylow]
	Es sei \(G\) eine endliche Gruppe der Ordnung $|G|=p^n q$ mit \(p\) Primzahl, \(q\) teilerfremd zu \(p\).
	Dann gelten:
	\begin{enumerate}
	\item \(G\) enthält eine Untergruppe der Ordnung $p^\alpha$ für jedes $1\leq \alpha \leq n$. % TODO Wieso nicht 0 \leq ...? Mit {e} ist das eine triviale Verstärkung des Satzes. -- Ich halte das ausnahmsweise für viel zu trivial. Sinn des Satzes ist ja die Existenz von nichttrivialen p-Gruppen zu zeigen. (Fabian)
	\item Es sei \(H\) eine \(p\)-Untergruppe von \(G\) und \(S\) eine \(p\)-Sylowuntergruppe.
		Dann existiert ein $g \in G$ mit $H \subset gSg^{-1}$.
	\item Die Anzahl $s$ der \(p\)-Sylowuntergruppen ist ein Teiler von \(q\) von der Form $s=1+kp$ mit $k \in \mathbb{N}$.
	\end{enumerate}
\end{thm}

\begin{proof}
\begin{enumerate}
\item Wir zeigen dies durch Induktion auf $\ord(G)$.
	Wir nehmen an, dass die Aussage für alle Gruppen gilt, deren Ordnung kleiner als \(\ord(G)\) ist.
	Wenn \(G\) eine echte Untergruppe \(H\) enthält, deren Ordnung durch \(p^n\) teilbar ist, existiert nach der Induktionsvoraussetzung eine Untergruppe von \(H\) von der Ordnung $p^\alpha$.
	Wir können uns also auf den Fall beschränken, dass $p^n \nmid \ord(H)$ für alle Untergruppen $H\neq G$ gilt.

	Ist $(a_i)_{i \in I}$ ein Repräsentantensystem der Konjugationsklassen, so gilt
	\[p^nq = |G| = |Z(G)|+\sum_{a_{i}\notin Z(G)}\dfrac{|G|}{|C_{G}(a_{i})|} = |Z(G)|+\sum_{a_{i}\notin Z(G)}[G:C_G(a_i)].\]
	Da die \(C_G(a_i)\) für alle \(a_i \not\in Z(G)\) echte Untergruppen sind, gilt $p^n \nmid |C_G(a_i)|$.
	Aus der Annahme $p^n \divides |G|$ ergibt sich also \(p \divides [G:C_G(a_i)]\), weshalb auch $|Z(G)|$ durch \(p\) teilbar ist.
	Aus dem Satz von Cauchy folgt dann, dass $b \in Z(G)$ mit $\ord(b)=p$ existiert.

	% VL 09.06.
	Da $b$ im Zentrum von $G$ liegt, ist $\langle b \rangle$ eine normale Untergruppe von $G$.
	Wir betrachten die Faktorgruppe $G/ \langle b \rangle$.
	Es gilt $$|G/\langle b\rangle| = \frac{|G|}{|\langle b\rangle|} = p^{n-1}q < p^nq.$$
        Nach Induktionsvoraussetzung enthält $G/\langle b\rangle$ eine Untergruppe $\bar{S}$ der Ordnung $p^{\alpha - 1}$. Es sei $S = \{x \in G : x \langle b \rangle \in \bar{S}\}$.\footnote{Es ist \(S = \pi^{-1}(\bar S)\) das Urbild der Untergruppe \(\bar S\) bzgl. der natürlichen Projektion \(\pi: G \to G/\langle b\rangle\). Eine weitere Konstruktionsmöglichkeit wäre \(S = \bigcup_{s \in\bar S} s\), wobei ein jeder sich für die am intuitivsten scheinende Variante entscheiden möge.} %FIXME Sollte es vielleicht \bigcup_{s \langle b \rangle \in \bar S} heißen? Zur einheitlichen Notation schlage ich vor das kleine s zusätzlich noch durch ein x zu ersetzten, so wie es auch im Beweis verwendet wird. (Fabian)
	$S$ ist eine Untergruppe von \(G\) und es gilt $\bar{S} \cong S/\langle b\rangle$, denn für die natürliche Projektion $p: S \rightarrow \bar{S}$ gilt $\ker(p) = \langle b\rangle$ und somit ist der Homomorphiesatz anwendbar.
	Das liefert $$p^{\alpha - 1} = |\bar{S}| = \frac{|S|}{| \langle b \rangle |} = \frac{|S|}{p}.$$
	Somit gilt $|S| = p^{\alpha}$ und $S$ ist die gesuchte $p$-Gruppe.
\item Sei $H$ eine $p$-Untergruppe und $S$ eine $p$-Sylowuntergruppe.
	$H$ operiert auf der Menge $X = G/S$ durch
	$$H \times X \rightarrow X, \quad (h, gS) \mapsto h \cdot gS := (hg)S.$$
	Es gilt $|X| = \frac{|G|}{|S|} = \frac{p^nq}{p^n} = q$.
	Da $H$ eine $p$-Gruppe ist, gilt $|X^H| = |X| = q \pmod{p}$. % FIXME: Wieso? Außerdem: "\equiv" statt "="?
	Da $q$ prim zu $p$ ist, ist $|X^H|$ nicht durch $p$ teilbar.
	Insbesondere kann $X^H$ nicht leer sein, denn $p \divides 0$.
	Es gibt also ein $g \in G$ mit $gS \in X^H$.
	Für alle $h \in H$ gilt dann $(hg)S = gS$ und somit $g^{-1}hg \in S$ und $h \in gSg^{-1}$.
	Dies liefert, dass $H$ in $gSg^{-1}$ enthalten ist.
\item Siehe unten. \qedhere
\end{enumerate}
\end{proof}

\begin{kor}
	Die $p$-Sylowuntergruppen bilden eine Klasse von konjugierten Untergruppen von $G$.
\end{kor}

\begin{proof}
	Sei $S$ ein $p$-Sylowuntergruppe mit \(|S| = p^n\).
	Dann ist $gSg^{-1}$ ein Untergruppe der Ordnung $p^n$ (für $g \in G$ beliebig), d.h.\ $gSg^{-1}$ ist eine $p$-Sylowuntergruppe.

	Es sei $S'$ ein beliebige \(p\)-Sylowuntergruppe von $G$.
	Aus 2. folgt: $S' \subseteq gSg^{-1}$ für ein $g \in G$.
	$|S'| = p^n = |gSg^{-1}|$ impliziert $S' = gSg^{-1}$.
\end{proof}

Bevor wir den dritten Teil des Sylow-Satzes beweisen, wenden wir uns folgender allgemeiner Frage zu:
Wie bestimmt man die Anzahl der zu einer gegebenen Untergruppe konjugierten Untergruppen?
Es sei $\mathcal{U}$ die Menge aller Untergruppen von \(G\) und $G \times \mathcal{U} \rightarrow \mathcal{U}, (g , H) \mapsto g \cdot H := gHg^{-1}$ eine Gruppenoperation.

\begin{defn}
	Der Stabilisator einer Untergruppe $H$ von $G$ unter dieser Operation heißt \emph{Normalisator} von $H$: $N_G(H) = \{ g \in G: gHg^{-1} = H \}$.
	Die zu $H$ konjugierten Untergruppen bilden die Bahn $G \cdot H$ von $H$.
	Es gilt also $$|G \cdot H| = | \{ gHg^{-1} \in \mathcal{U} : g \in G \} | = [G:N_G(H)].$$
	Für eine endliche Gruppe $G$ ist diese Anzahl immer ein Teiler von $|G|$.
\end{defn}

Für den Beweis des dritten Teils benötigen wir folgendes Lemma:

\begin{lemma}\label{lem:Sylow}
	Sei $T$ eine $p$-Sylowuntergruppe von $G$ und $H$ eine $p$-Untergruppe. Wenn $H \subseteq N_G(T)$ gilt, gilt bereits $H \subseteq T$.
\end{lemma}

\begin{proof}
	Da $T$ eine normale Untergruppe von $N_G(T)$ ist und $H$ eine Untergruppe von $N_G(T)$ ist, kann man den ersten Isomorphiesatz anwenden: $TH = \{ gh : g \in T, h \in H \}$ ist eine Untergruppe von $N_G(T)$ und es gilt $H/(T \cap H) \cong TH/T$. Daraus folgt
	$$ [TH : T] = [H : T \cap H] = \frac{|H|}{|T \cap H|} = \frac{p^{\alpha}}{|T \cap H|} = p^{\alpha '} \text{ mit } 0 \leq \alpha ' \leq \alpha.$$
	$TH/T$ ist also eine $p$-Gruppe. Es gilt: $|TH/T| = 1 \Leftrightarrow T \cap H = H \Leftrightarrow H \subseteq T$. \\
	Die Inklusionskette $T \subseteq TH \subseteq G$ von Untergruppen ergibt
	$$[G:T] = [G:TH][TH:T] \Leftrightarrow q = [G:TH]p^{\alpha '} \text{, d.h. } p^{\alpha '} \divides q.$$
	Da $q$ nicht durch $p$ teilbar ist, muss $p^{\alpha '} = 1$ sein. Also ist $H$ in $T$ enthalten.
\end{proof}

\begin{proof}[Beweis des dritten Teils des Sylow-Satzes]
	Sei $X$ die Menge aller $p$-Sylowuntergruppen von $G$, $S \in X$ beliebig (Existenz: siehe Teil 1), $s := |X|$. Es gilt $s = [G:N_G(S)]$. \\
	Aus der Inklusionskette $S \subseteq N_G(S) \subseteq G$ folgt
	$$[N_G(S):S] [G:N_G(S)] = [G:S] \Leftrightarrow [N_G(S):S] s = q \text{, d.h. } s \divides q.$$
	Wir lassen $S$ wie folgt auf $X$ operieren:
	$$S \times X \rightarrow X, (g, T) \mapsto gTg^{-1}.$$
	$S \in X$ ist ein Fixpunkt dieser Operation. Wir möchten zeigen, dass dieser der einzige Fixpunkt ist. Da $S$ eine $p$-Gruppe ist, gilt dann: $s = |X| = |X^S| = 1 \pmod{p}$ (was zu zeigen war).
	\begin{align*}
		T \in X^S &\Leftrightarrow &gTg^{-1} = T \text{ für alle } g \in S \\
		&\Leftrightarrow &g \in N_G(T) \text{ für alle } g \in S \\
		&\Leftrightarrow &S \subseteq N_G(T)
	\end{align*}
	Aus $S \subseteq N_G(T)$ folgt mit Lemma~\ref{lem:Sylow}, dass $S \subseteq T$ und somit $S = T$ (da $|S| = |T| = p^n$). $S \in X$ ist damit der einzige Fixpunkt.
\end{proof}

\underline{Anwendungen/Folgerungen des Sylow-Satzes}

\begin{thm}
  Es seien $p, q$ Primzahlen mit $p < q$ und $p \nmid q-1$.\footnote{Dies schließt gerade den Fall $p = 2, q = 3$ aus. Ein Isomorphietyp von Gruppen der Ordnung 6 ist nämlich $D_6$ und diese Gruppe ist nicht zyklisch. S.a. Beispiel \ref{ex:semidirekt}.} Dann ist jede Gruppe $G$ der Ordnung $pq$ eine zyklische Gruppe.
\end{thm}

\begin{proof}
	Es sei $s$ die Anzahl der $p$-Sylowuntergruppen von $G$. Aus Teil 3 des Sylow-Satzes folgt $s \divides q$ und $s \equiv 1 \pmod{p}$. Da $q \not \equiv 1 \pmod{p}$, muss $s = 1$ gelten, denn die Primzahl $q$ ist nur durch $1$ und $q$ teilbar. 

	Es gibt also eine einzige $p$-Sylowuntergruppe $S$ von $G$. Aus Teil 2 des Sylow-Satzes folgt $gSg^{-1} = S$ für alle $g \in G$. Somit ist $S$ eine normale Untergruppe. %FIXME Wozu den zweiten Sylowsatz. Die Aussage sollte eigentlich direkt daraus folgen, dass die Konjugation als ein Isomorphismus die Gruppenordnung unverändert lässt. Eventueller Fix betrifft dann auch die Aussage im thm. zur einfachen Gruppe der Ordnung 30
	Für die Anzahl $r$ der $q$-Sylowuntergruppen gilt: $r \divides p$ und $r \equiv 1 \pmod{q}$. Wenn $r = p$, dann gilt $r = p \not = 1 \pmod{q}$ (wegen $p < q$). Dies ist also nicht möglich. Es gilt dann $r = 1$. Es existiert genau eine $q$-Sylowuntergruppe $R$ von $G$ und $R$ ist deswegen ein Normalteiler von $G$. Da $S$ und $R$ normal sind, ist $SR = \{ ab : a \in S, b \in R\}$ eine normale Untergruppe. 

	Es gilt $S \cap R = \{ e\}$ (die Elemente müssen Ordnung $1$ haben). Dies ergibt, dass $SR$ das (innere) direkte Produkt von $S$ und $R$ ist. Insbesondere gilt $|SR| = |S \times R| = |S||R| = pq = |G|$. Folglich gilt $G = SR \cong S \times R$. 

        Wir zeigen nun $G = \langle ab \rangle$. Dazu sei $S =  \langle a \rangle$ und $R =  \langle b \rangle$. Es gilt $ (ab)^{pq} = (a^p)^q (b^q)^p = e$ und damit $\ord(ab) \divides pq$. Da $p$ und $q$ teilerfremde Primzahlen sind folgt $ (ab)^p = a^p b^p = b^p \neq e$ und  $(ab)^q = a^q b^q = a^q \neq e$. Also gilt $\ord(ab) = pq$.
\end{proof}

% VL 16.06.
\begin{thm}
	Es existiert keine einfache Gruppe der Ordnung 30.
\end{thm}

\begin{proof}
	Es ist $|G| = 30 = 2\cdot 3\cdot 5$. Es seien $s$ die Anzahl der 5-Sylowuntergruppen und $t$ die der 3-Sylowuntergruppen. Angenommen, $G$ sei einfach. Dann gilt: $s>1$ und $t>1$, denn mit $s=1$ (bzw. $t=1$) wäre die 5-Sylowuntergruppe (bzw. 3-Sylowuntergruppe) normal wegen Konjugiertheit der Sylowgruppen zueinander (Teil 2 im Satz). Aus Teil (3) folgt
		\[ s \divides 6,\; s\equiv 1(5)\Rightarrow s=6.\]
	Also gibt es 6 verschiedene 5-Sylowuntergruppen, jede davon hat Ordnung $5$. Da die 5-Sylowuntergruppen Primzahlordnung haben und damit insbesondere zyklisch sind, haben sie paarweise nur einen trivialen Schnitt. Damit enthalten die 5-Sylowuntergruppen zusammen $6\cdot 4 =24$ Elemente und das neutrale Element. Analog gilt für $t$
		\[ t \divides 10,\; t\equiv 1(3) \Rightarrow t=10.\]
	Damit enthalten die 3-Sylowuntergruppen $10\cdot 2=20$ Elemente und das neutrale Element. Die nichttrivialen Elemente der 3-Sylowuntergruppen sind von denen der 5-Sylowuntergruppen verschieden, denn das einzige Element, dessen Ordnung sowohl 5 als auch 3 teilt, ist das neutrale.
	
	Aber: $20+24+1=45>30=\ord(G)$. Also kann $G$ nicht einfach sein.
\end{proof}

\subsection{Normal- und Kompositionsreihen}

\begin{defn}
	Es sei $G$ eine Gruppe. Eine Normalreihe in $G$ ist eine Folge von Untergruppen
		\[ \{e\} = G_0 \subseteq G_1 \subseteq G_2 \subseteq\dots\subseteq G_n=G\qquad n\geq 1\qquad(n=0\text{ für } G=\{e\}),\]
	sodass für alle $i$ mit $1\leq i\leq n$ die Untergruppe $G_{i-1}$ Normalteiler von $G_i$ ist.
\end{defn}

\begin{nb}
	Die $G_i,\;i\leq n-2,$ sind nicht notwendigerweise Normalteiler von $G$.
\end{nb}

\begin{defn}
	Es seien 
		\[\begin{split}
			&\underline{G}: G_0\subseteq G_1 \subseteq\dots\subseteq G_n \\
			&\underline{H}: H_0\subseteq H_1\subseteq\dots\subseteq H_m 
		\end{split}\]
	zwei Normalreihen.
	\begin{itemize}
		\item $\underline{H}$ ist eine Verfeinerung von $\underline{G}$, wenn $\underline{H}$ aus $\underline{G}$ durch das Einfügen von Untergruppen erhalten werden kann.
		\item $\underline{G}$ und $\underline{H}$ heißen äquivalent, wenn $m=n$ gilt und es eine Permutation $\sigma\in S_n$ gibt, sodass 
		\[ G_j/G_{j-1}\cong H_{\sigma(j)}/H_{\sigma(j)-1}\]
			für alle $1\leq j\leq n$ gilt.
	\end{itemize}
\end{defn}

\begin{thm}
Es sei $G$ eine Gruppe, $\underline{G}$ und $\underline{H}$ Normalreihen in $G$. Dann besitzen $\underline{G}$ und $\underline{H}$ äquivalente Verfeinerungen.
\end{thm}

\begin{proof}
	Es seien 
	\[\begin{split}
			&\underline{G}: G_0\subseteq G_1 \subseteq\dots\subseteq G_n \\
			&\underline{H}: H_0\subseteq H_1\subseteq\dots\subseteq H_m 
		\end{split}\]
	die beiden Normalreihen. Wir setzen nun $G_{ij}:=G_i\cdot(G_{i+1}\cap H_j), H_{ij} := H_j\cdot(G_i\cap H_{j+1})$ für alle $i,j$ mit $G_{n+1}:=G=:H_{m+1}$.
	
	Es gilt nun: $G_{i,m}=G_{i+1}=G_{i+1,0}$ und $H_{n,j}=H_{j+1}=H_{0,j+1}$, sodass $\{G_{ij}\}$ und $\{H_{ij}\}$ Folgen von Untergruppen sind. Es ist klar, dass diese Verfeinerungen von $G$ bzw. $H$ von selber Länge sind. Wir müssen noch zeigen:
	\begin{enumerate}[label=(\alph*)]
		\item $G_{ij}\subseteq G_{i,j+1}$ und $H_{ij}\subseteq H_{i,j+1}$ sind normal, damit beide Normalreihen sind,
		\item $G_{i,j+1}/G_{i,j}$ und $H_{i+1,j}/H_{i,j}$ sind beide isomorph zu $\nicefrac{(G_{i+1}\cap H_{j+1})}{(G_{i+1}\cap H_{j})\cdot(G_{i}\cap H_{j+1})}$.
	\end{enumerate}
	Dann sind die beiden Verfeinerungen äquivalent. Beweis von (a) und (b) erfolgt mit Anwendung der Isomorphiesätze.
\end{proof}

\begin{defn}
	Eine Normalreihe 
		\[\{e\}=G_0\subsetneqq G_1 \subsetneqq\dots\subsetneqq G_{n-1}\subsetneqq G_n=G \]
	heißt Kompositionsreihe, wenn sie keine echten Verfeinerungen besitzt.
\end{defn}

\begin{kor}[Satz von Jordan-Hölder]
	Es sei $G$ eine Gruppe. Zwei Kompositionsreihen in $G$ sind immer zueinander äquivalent.
\end{kor}

\begin{thm}
	Eine Normalreihe ist genau dann eine Kompositionsreihe, wenn die $G_i/G_{i-1}$ jeweils einfache Gruppen sind.
\end{thm}

\begin{proof}
	"$\Rightarrow$": Angenommen, $G_i/G_{i-1}$ sei für ein $i\leq n$ nicht einfach. Es sei dann $\bar{N}\subset G_i/G_{i-1}$ eine nicht triviale normale Untergruppe. Nun setzen wir 
		\[N:=\{g\in G_i: gG_{i-1}\in\bar{N}\}.\]
	Dann ist $N$ eine normale Untergruppe mit $\nicefrac{N}{G_{i-1}}=\bar{N}$. Aus $\{eG_{i-1}\}\subsetneqq\bar{N}\subsetneqq\nicefrac{G_i}{G_{i-1}}$ folgt $G_{i-1}\subsetneqq N\subsetneqq G_i$. Somit können wir $\{G_i\}$ mit $N$ verfeinern, also ist $\{G_i\}$ keine Kompositionsreihe.
	
	"$\Leftarrow$": Wenn $G_0\subsetneqq G_1\subsetneqq\dots\subsetneqq G_n$ eine echte Verfeinerung besitzt, so gibt es eine normale Untergruppe $G_{i-1}\subsetneqq N\subsetneqq G_i$, die $G_{i-1}$ als Normalteiler enthält. Die Faktorgruppe $\bar{N}:=\nicefrac{N}{G_{i-1}}$ ist dann eine nicht-triviale normale Untergruppe von $\nicefrac{G_i}{G_{i-1}}$. Also ist $G_i/G_{i-1}$ nicht einfach.
\end{proof}

\begin{thm}
	Jede endliche Gruppe besitzt eine Kompositionsreihe.
\end{thm}

\begin{proof}
	Übungsaufgabe.
\end{proof}

\begin{ex}

	\begin{itemize}
		\item Unendliche Gruppen besitzen nicht immer Kompositionsreihen. So hat $(\mathbb{Z},+)$ keine Kompositionsreihe, da man jede Normalreihe beliebig verfeinern kann.
		\item Jede symmetrische Gruppe $S_n,\; n\geq 3$ hat die Normalreihe
			\[\{e\}\subsetneqq A_n \subsetneqq S_n.\]
			In dieser Normalreihe haben wir die Faktorgruppen: $\nicefrac{S_n}{A_n}\cong\nicefrac{\mathbb{Z}}{2\mathbb{Z}}$ und $\nicefrac{A_n}{\{e\}}\cong A_n$. Nun ist $\nicefrac{\mathbb{Z}}{2\mathbb{Z}}$ einfach, da die Ordnung eine Primzahl ist, und für $n\neq 4$ ist $A_n$ einfach. Also ist die obige Normalreihe für diese Fälle eine Kompositionsreihe.
	\end{itemize}
\end{ex}

\begin{kor}
	Für $n\geq 3,n\neq 4$ ist $A_n$ die einzige Untergruppe von $S_n$ vom Index $2$. Damit ist $\{e\}\subset A_n\subset S_n$ die einzige Kompositionsreihe und $A_n$ die einzige nicht-triviale normale Untergruppe von $S_n$.
\end{kor}

\subsection{Auflösbare Gruppen}

\begin{defn}
	Eine Gruppe $G$ heißt auflösbar, wenn sie eine abelsche Normalreihe, das heißt eine Normalreihe mit abelschen Faktorgruppen besitzt.
\end{defn}

\begin{ex}
	\begin{itemize}
		\item Abelsche Gruppen sind immer auflösbar.
		\item Die Gruppe $G=\left\{\begin{pmatrix}
			a & b \\
			c & d \end{pmatrix} \in \text{GL}_2(K):c=0\right\}$ ist für jeden Körper $K$ auflösbar.
	\end{itemize}
\end{ex}

\begin{thm}
	Es sei $G$ eine Gruppe.
	\begin{itemize}
		\item Ist $G$ auflösbar, so ist jede Untergruppe von $G$ auflösbar.
		\item Ist $N$ eine normale Untergruppe von $G$, so ist $G$ genau dann auflösbar, wenn $N$ und $\nicefrac{G}{N}$ auflösbar sind.
	\end{itemize}
\end{thm}

\begin{proof}
	(Idee: $H\subset G$ Untergruppe, $G_0\subset G_1\subset\dots\subset G_n$ abelsche Normalreihe für $G$. Dann ist $(G_0\cap H)\subset(G_1\cap H)\subset\dots\subset(G_n\cap H)$ eine abelsche Normalreihe in $H$.)
\end{proof}

\begin{ex}
	Die symmetrische Gruppe $S_n$ ist auflösbar für $n\leq 4$. Die Fälle $n\leq 2$ sind sogar abelsch. Für $S_3$ erhalten wir die Kompositionsreihe
		\[\{e\}\subset A_3=\langle(123)\rangle\subset S_3\]
	mit Faktoren $\nicefrac{S_3}{A_3}\cong\nicefrac{\mathbb{Z}}{2\mathbb{Z}}$, $A_3\cong\nicefrac{\mathbb{Z}}{3\mathbb{Z}}$, die beide zyklisch und somit abelsch sind.
	
	Die Gruppe $S_4$ besitzt die Normalreihe
		\[\{e\}\subset V_4\subset A_4\subset S_4\]
	mit $V_4=\left\{\tau\in A_4:\tau^2=e\right\}=\{e,(12)(34),(13)(24),(14)(23)\}$, der kleinschen Vierergruppe. Dann ist $V_4\cong\nicefrac{\mathbb{Z}}{2\mathbb{Z}}\times\nicefrac{\mathbb{Z}}{2\mathbb{Z}}$ abelsch (denn die Ordnung ist $4=2^2=p^2$ für eine Primzahl $p$) und $\nicefrac{A_4}{V_4}\cong\nicefrac{\mathbb{Z}}{3\mathbb{Z}}$, $\nicefrac{S_4}{A_4}\cong\nicefrac{\mathbb{Z}}{2\mathbb{Z}}$ sind auch abelsch.
\end{ex}

\begin{thm}
	Die symmetrische Gruppe $S_n,\;n\geq 5$ ist nicht auflösbar.
\end{thm}

\begin{proof}
	Die Gruppe $A_n$ ist nicht auflösbar, da sie eine nicht abelsche einfache Gruppe ist. Da $A_n$ eine Untergruppe von $S_n$ ist, ist $S_n$ auch nicht auflösbar.
\end{proof}

%VL 23.06.15 Fabian Gabel

Motivation: Beschreibung von Nullstellen von Polynomen durch Radikale
\begin{itemize}
  \item Grad $2$: $x^2 + 2px + q = 0 \leadsto p\text{-}q\text{-Formel} \quad x_{1,2} = -p \pm \sqrt{p^{2} - q} $
  \item Grad $3,4$: Ähnliche Formeln, die $\sqrt[2]{}, \sqrt[3]{}, \sqrt[4]{}$ enthalten
\end{itemize}

Aus der Tatsache, dass $S_n$ nicht auflösbar ist für $n \geq 5$ folgt, dass solche Formeln für allgemeine Polynome vom Grad $n \geq 5$ nicht mehr möglich sind.

\subsection{Abelsche Gruppen}

Im Folgenden wird die Notation $(G,+)$ -- additive Verknüpfung -- für abelsche Gruppen verwendet.

\underline{Wiederholung}: $(H_i)_{i \in I}$ Familie von abelschen Gruppen.

Direktes Produkt: $\prod_{i \in I} H_i \to (h_i)_{i \in I}$ mit $h_i \in H_i \; \forall i \in I$.

\begin{defn}
  Die \emph{(äußere) direkte Summe} der Familie $(H_i)_{i \in I}$ von abelschen Gruppen ist 
  \[ \bigoplus_{i \in I} H_i := \left\{ (h_i)_{i \in I} \in \prod_{i \in I} H_i : h_i \neq 0 \text{ für nur endlich viele $i \in I$}\right\}. \]
\end{defn}

\begin{nb}
  $\bigoplus_{i \in I} H_i$ ist eine Untergruppe von $\prod_{i \in I} H_i$.
\end{nb}

Wie im Fall der direkten Produkte zeigt man das Folgende:

\begin{thm}
  Es sei $(H_i)_{i \in I}$ eine Familie von Untergruppen einer abelschen Gruppe $G$. Dann sind die folgenden Aussagen äquivalent:
\begin{itemize}
  \item[(i)]  $G$ wird durch $\cup_{i \in I} H_i$ erzeugt und für jedes $i_0 \in I$ gilt 
    \[ H_{i_0} \cap \langle \bigcup_{i \neq i_0} H_i \rangle = \left\{ 0 \right\}. \]
  \item Jedes $g \in G$ besitzt eine eindeutige Darstellung $g = h_{i_1} + \dots + h_{i_l}$ mit $h_{i_j} \in H_{i_j}$ für $j = 1,\dots,l$, ($i_1,i_2,\dots,i_l$ paarweise verschieden).
\end{itemize}
\end{thm}

\begin{nb}
  $ \bigoplus_{i \in I} H_i = \prod_{i \in I} H_i $ für I endlich.
\end{nb}

\begin{nb}
  Wenn diese Aussagen erfüllt sind, ist $G$ die \emph{(innere) direkte Summe} von $(H_i)_{i \in I}$. Aus (ii) folgt, dass $G$ zu $\bigoplus_{i \in I} H_i$ isomorph ist. 
\end{nb}

\begin{defn}
  Eine abelsche Gruppe heißt \emph{frei}, wenn eine Teilmenge $X \subset G$ existiert, sodass
  \begin{itemize}
    \item[(i)] $\operatorname{ord}(x) = \infty$ gilt für alle $x \in X$
    \item[(ii)] $G = \bigoplus_{x \in X} \langle x \rangle$.
  \end{itemize}

  Die Menge $X$ heißt dann eine \emph{Basis} von $G$ (nicht eindeutig bestimmt). Ist $G$ frei mit Basis $X$, so besitzt jedes $g \in G$ eine eindeutige Darstellung $g = \sum_{x \in X} m_x x$ mit $m_x \in \mathbb{Z}$ und $m_x = 0$ für fast alle $x \in X$ (d.h. bis auf eine endliche Menge).
\end{defn}

\begin{ex}
  \begin{itemize}
    \item $G = \bigoplus_{i \in I} H_i \quad H_i \cong \mathbb{Z}$
    \item $M$ feste Menge, $G = \left\{ f : M \to \mathbb{Z} \divides f(m) = 0 \text{ für fast alle } m \in M \right\}$ ist freie abelsche Gruppe.  Für $m \in M$ definiert man $f_m \in G$ durch $f_m(m) = 1, f_m(m') = 0$ für $m' \in M, m' \neq m$. $X = \left\{ f_m \divides m \in M \right\}$ ist dann eine Basis von $G$.
  \end{itemize}
\end{ex}

\begin{nb}
  Es sei $G$ eine abelsche Gruppe, die ein $b \neq 0 $ mit $\operatorname{ord}(b) < \infty$ enthält. Dann ist $G$ \emph{nicht} frei abelsch.
\end{nb}

\begin{proof}
  Es sei $G$ frei abelsch, $g \in G$ mit $ 1 < \operatorname{ord}(g) = n < \infty$. Dann gilt:
  \begin{itemize}
    \item $g = \sum_{x \in X} m_x x $ für $X$ Basis von $G$ 
    \item $0 = ng = \sum_{x \in X} (n m_x) x $
    \item[Aber:] $0 = \sum_{x \in X} 0 \cdot x$
  \end{itemize}
  Dies liefert $n m_x = 0$ für alle $x \in X$. Daraus folgt (wegen $n \neq 0$) $m_x = 0$ für alle $x$. Daraus folgt: $g = 0$.
\end{proof}

Freie abelsche Gruppen genießen die folgende universelle Eigenschaft:  

\begin{thm}
  Sei $G$ eine freie abelsche Gruppe mit Basis $X$ und $H$ eine beliebige abelsche Gruppe. Sei $f : X \to H$ eine Abbildung, Dann existiert ein eindeutig bestimmter Homomorphismus $\tilde{f}: G \to H$, der $f$ fortsetzt, d.h. mit $\tilde{f}(x) = f(x)$ für alle $x \in X$.
\end{thm}
\begin{proof}
  Es sei $g \in G$ beliebig. Es gilt: $g = \sum_{x \in X}m_x x$ ($ m_x = 0$ für fast alle $x$).
  Wir setzen $\tilde{f}(g) := \sum_{x \in X} m_x f(x)$ (Summe ist wieder endlich, insbesondere unabhängig von Summationsreihenfolge). Dies ist wohldefiniert, weil die Summe endlich ist. $\tilde{f}$ ist dann ein Gruppenhomomorphismus, der $f$ fortsetzt. %FIXME Eigentlich folgt die Wohldefiniertheit daraus, dass jedes g eine EINDEUTIGE Darstellung als Z-Linearkombination von Basiselementen besitzt. Dies beinhaltet zudem die benötigte Endlichkeit der Summe (Fabian)

  Es sei $\varphi : G \to H$ ein Homomorphismus, der $f$ fortsetzt. Dann gilt für alle $x_1,\dots,x_l \in X, m_1,\dots,m_l \in \mathbb{Z}$: 
  \begin{align*}
  \varphi(m_1 x_1 + \dots + m_l x_l) 
  &= \varphi( m_1 x_1) + \dots \varphi( m_l x_l ) \\
  &= m_1 \varphi(x_1) + m_2 \varphi(x_2) + \dots + m_l \varphi(x_l) \\
  &= m_1 f(x_1) + m_2 f(x_2) + \dots + m_l f(x_l) \\
  &= m_1 \tilde{f}(x_1) + \dots + m_l \tilde{f}(x_l)\\
  &= \tilde{f}( m_1 x_1 + \dots m_l x_l)
  \end{align*}
  Dies zeigt, dass $\varphi(g) = \tilde{f}(g) $ für beliebige $g \in G$ gilt. Damit ist die Fortsetzung eindeutig.
\end{proof} 

\begin{kor}
  Jede abelsche Gruppe $G$ ist (isomorph zu einer) Faktorguppe einer freien abelschen Gruppe. D.h. zu jeder abelschen Gruppe $G$ gibt es eine freie abelsche Gruppe $F$ mit einem surjektiven Homomorphismus $f: F \to G$.
\end{kor}

\begin{proof}
  Für jedes $g \in G$ wählen wir eine Gruppe $H_g \cong \mathbb{Z}$ mit Erzeuger $x_g \in H_g$ und setzen:
  \[ F := \bigoplus_{g \in G} H_g \cong \bigoplus_{g \in G} \mathbb{Z} \]
  $F$ ist dann eine freie abelsche Gruppe mit Basis $X = \left\{  x_g \,\vert\, g \in G \right\}$.

  Wir betrachten $f : X \to G, x_g \mapsto g$. Diese Abbildung $f$ wird nach dem obigen Satz durch einen Homomorphismus $\tilde{f}: F\to G$ fortgesetzt, der offensichtlich surjektiv ist. 
  Unter Verwendung des Homomorphiesatzes folgt sogleich $F / \ker(\tilde f) \cong G$.
\end{proof}

\begin{thm}
  Seien $G_1, G_2$ freie abelsche Gruppen mit Basen $X_1 \subset G_1, X_2 \subset G_2$. Dann gilt:
  \[ G_1 \cong G_2 \iff \lvert X_1 \rvert = \lvert X_2 \rvert \]
\end{thm}
\begin{proof}
  "$\Leftarrow$": $\lvert X_1 \rvert = \lvert X_2 \rvert$ bedeutet, dass es eine Bijektion $f: X_1 \to X_2$ gibt. Es sei $\tilde{f}: G_1 \to G_2 $ der Gruppenhomomorphismus der $f : X_1 \to X_2 \subset G_2$ fortsetzt.
  Ähnlich existiert ein Homomorphismus $\tilde{g}: G_2 \to G_1$, der $f^{-1}$ fortsetzt. Da $(\tilde{g} \circ \tilde{f})(x) = (f^{-1} \circ f)(x) = x $ für alle $x \in X_1$ gilt, ist $\tilde{g} \circ \tilde{f} $ eine Fortsetzung von $\operatorname{id}_{X_1} : X_1 \to X_1$. Aus der Eindeutigkeit folgt dann, dass $\tilde{g} \circ \tilde{f} = \operatorname{id}_{G_1}$. Genauso zeigt man: $\tilde{f} \circ \tilde{g} = \operatorname{id}_{G_2}$. Da $\tilde{f}$ das Inverse $\tilde{g}$ besitzt, ist $\tilde{f}: G_1 \to G_2$ dann ein Isomorphismus.

  "$\Rightarrow$": Es sei $p $ eine Primzahl, $G$ eine freie abelsche Gruppe. 
  $pG = \left\{ pg \,\vert\, g \in G \right\}$ ist dann eine Untergruppe von $G$. 
  Dann ist $V = G / pG$ ein Vektorraum über $\mathbb{F}_p = \mathbb{Z}/p\mathbb{Z}$. 
  Sei $G \to G/pG = V, g\mapsto \overline{g} = g + pG$  die natürliche Projektion. 
  Wenn $G$ eine freie abelsche Gruppe mit Basis $X$ ist, dann ist $ \overline{X} = \left\{ \overline{x} \,\vert\, x \in X \right\} $ eine Basis vom Vektorraum $V$. 
  Nun ist die Abbildung $X \to \overline{X}, x \mapsto \overline{x} = x + pG$ ist eine Bijektion: Sie ist offensichtlich surjektiv; 
  für Injektivität nehme man an, es existieren $x, x' \in X$ mit $x \neq x'$ und $\overline{x} = \overline{x'}$. 
  Daraus folgt $ x - x' \in pG$, was wiederum $x - x' = p \, ( \sum_{y \in X} m_y y )  = \sum_{y \in X} \left( p m_y \right) y $ impliziert. 
  Aus der Eindeutigkeit der Darstellung in $G$ folgt $p m_x = 1$, bzw. $p m_{x'} = -1$ wobei $m_x, m_{x'} \in \mathbb{Z}$ $\leadsto$  Widerspruch. Somit ist die Abbildung auch injektiv.
  %@Sebastian: FIXME Hier fehlt noch der Nachweis der Basiseigenschaft, d.h. Lineare U. und Erzeugendensystem, der am Anfang der Vorlesung vom 30.06. erbracht wurde (Fabian)
\end{proof}

%Vorlesung 30.06.15



%Vorlesung 7.7.15

\begin{nb}
Ziel ist die Klassifizierung der endlich erzeugten abelschen Gruppen:
\begin{enumerate}
\item Schritt: Zeige $G \cong F \oplus T(G)$ mit $F$ endlich erzeugter freier abelscher Gruppe (also $F \cong \mathbb{Z}^{r}, r \geq 1$) und der endlichen Torsionsuntergruppe $T(G)$ von G.
\item Schritt: Zeige (mit Sylow) die Zerlegung $T(G) = \bigoplus_{p\text{ prim}} G_{p}$ mit geeigneten $p$-Gruppen $G_{p}$.
\item Schritt: Klassifizierung der $p$-Gruppen $G_{p}$, diese sind aus endlichen Produkten (endlicher) zyklischer Gruppen aufgebaut.
\end{enumerate}
\end{nb}

\begin{thm}
Sei $G$ eine torsionsfreie endlich erzeugte abelsche Gruppe. Dann ist $G$ frei abelsch.
\end{thm}

\begin{proof}
Sei $M$ ein endliches Erzeugendensystem von $G$. 
Sei $\{m_{1}, \dots ,m_{k}\} \subset M$ eine maximale Teilmenge $\mathbb{Z}$-linear unabhängiger Elemente, d.h. aus $n_{1}m_{1}+ \dots +n_{k}m_{k}=0$, mit $n_{1}, \dots ,n_{k} \in \mathbb{Z}$ folgt immer $n_{1}= \dots =n_{k}=0$. 
Folglich gibt es zu jedem $m \in M, m \notin \{m_{1}, \dots ,m_{k}\}$ ein $n(m) \in \mathbb{Z}, n(m)\neq 0$ und geeignete Koeffizienten $n_{1}, \dots ,n_{k} \in \mathbb{Z}$ mit $n(m)m=n_{1}m_{1}+ \dots +n_{k}m_{k}$.\footnote{Man betrachte hierzu eine aufgrund der Maximalität des Systems $\{m_1,\dots,m_k\}$ existierende nichttriviale $\mathbb{Z}$-Linearkombination der $0 \in G$ durch die Elemente aus $\{m,m_1,\dots,m_k\}$.}
Sei $N$ ein gemeinsames Vielfaches von den (endlich vielen) Zahlen $n(m),m \in M-\{m_{1}, \dots ,m_{k}\}$. 
Dann gilt: $Nm \in \langle m_{1}, \dots ,m_{k} \rangle$ für alle $m \in M$.

Sei $\mu :G\rightarrow \langle m_{1}, \dots ,m_{k} \rangle $ durch $\mu (g)=Ng$ definiert. Da $G$ abelsch ist, ist $\mu$ ein Homomorphismus. Wegen $\ker(\mu ) \subset T(G) = \{ 0\}$ ist $\mu$ injektiv. Aus dem Homomorphiesatz folgt $G \cong \mu (G) \subset \langle m_{1}, \dots ,m_{k}\rangle$. D.h. $G$ ist zu einer Untergruppe einer freien abelschen Gruppe isomorph. Somit ist nach Satz \ref{} $G$ frei abelsch. %@Sebastian: FIXME Der referenzierte Satz wurde noch nicht getext (VL 30.06.).
\end{proof}

\begin{kor}
Jede endlich erzeugte abelsche Gruppe ist eine direkte Summe einer freien abelschen Gruppe und einer endlichen Gruppe ($G=F\oplus T(G)$).
\end{kor}

\begin{proof}
  $G/T(G)$ ist torsionsfrei nach Lemma \ref{} und endlich erzeugt, also frei abelsch. Dann ist $T(G)$ nach Korollar \ref{} ein direkter Summand: $G = T(G) \oplus F,F\cong G/T(G)$. %@Sebastian: FIXME Das referenzierte Lemma und das referenzierte Korollar wurden noch nicht getext (VL 30.06.)

$T(G)$ ist eine Torsionsuntergruppe, die endlich erzeugt ist. Daraus folgt: 

$T(G)=\langle g_{1}, \dots ,g_{m}\rangle=\{ n_{1}g_{1}+ \dots +n_{m}g_{m}:n_{j}\in \mathbb{Z}\}=\{ n_{1}g_{1}+\dots+n_{m}g_{m}:|n_{j}| \leq \ord(g_{j})\} \Rightarrow |T(G)| < \infty$ %einfache Striche statt doppelten für Betrag und Gruppenordnung
\end{proof}

Endlich erzeugte freie abelsche Gruppen sind bekannt: Sie sind von der Form $F\cong \mathbb{Z}^{r}=\mathbb{Z} \oplus \dots\oplus \mathbb{Z}$, mit $r\geq 0$.

Man sollte daher auch die endlichen abelschen Gruppen klassifizieren.

\begin{defn}
Sei $G$ eine abelsche Gruppe und $p$ eine Primzahl. 
Dann heißt $G_p=\{ g\in G:p^{n}g=O \text{ für } n \in \mathbb{N} \} $ 
die \emph{$p$-primäre Komponente} von $G$.
\end{defn}

\begin{nb}
\begin{itemize}
\item $G_p\subset G$ ist eine Untergruppe (da $G$ abelsch).
\item Die Elemente von $G_p$ sind genau die $g\in G$, deren Ordnung eine Potenz von $p$ ist.
\item Alle $p$-Untergruppen von $G$ sind in $G_p$ enthalten, d.h. $G_p$ ist die einzige $p$-Sylowuntergruppe von $G$.
\end{itemize}
\end{nb}

\begin{thm}
Für $G$ endliche abelsche Gruppe gilt:
\[ G=\bigoplus_{p \text{ prim}} G_{p} \]
Die Zerlegung ist bis auf die Reihenfolge der Summanden eindeutig.
\end{thm}

\begin{proof}
Sei $| G| =p_{1}^{m_{1}}\dots p_{r}^{m_{r}}$ die Primfaktorzerlegung von $| G|$. 
Da die Ordnung eines Elements von $G$ immer $| G|$ teilt, gilt $G_p=\{ 0\}$ für alle $p$ prim $p\notin \{ p_{1},\dots,p_{r}\}$. Wir möchten zeigen:
\begin{equation} \label{eq}
G_{p_{i}} \,\cap\, \langle\bigcup_{j\neq i} G_{p_{j}} \rangle = \{ 0 \}
\end{equation}
Die Elemente von $\langle\bigcup_{j\neq i} G_{p_{j}}\rangle$ sind von der Form $g=\sum_{j\neq i} g_{j}, g_{j} \in G_{p_{j}}$. Damit folgt unter Beachtung von $\ord(g_{j})=p_{j}^{\alpha_{j}}, \alpha_{j}\geq 0$ 
\[(\prod_{j\neq i} p_{j}^{\alpha_{j}})g=0 \rightarrow \ord(g) \divides (\prod_{j\neq i} p_{j}^{\alpha_{j}}).\]
Gilt zusätzlich $g \in G_{p_{i}}$, so erhalten wir:
\[ \ord(g)=p_{i}^{\alpha_{i}} \divides (\prod_{j\neq i} p_{j}^{\alpha_{j}})\Rightarrow p_{i}^{\alpha_{i}}=1 \Rightarrow \ord(g)=1.\]
Aus \eqref{eq} folgt 
$$\langle\bigcup_{1\leq i\leq r} G_{p_{i}}\rangle = \underbrace{\bigoplus_{1 \leq i \leq r} G_{p_i}}_{\text{Ordnung: } |G_{p_1}|\cdot|G_{p_2}|\cdots|G_{p_r}| = p_1^{m_1}\cdots p_r^{m_r} = |G|} \subseteq G$$
und somit $\bigoplus_{1 \leq i \leq r} G_{p_i} = G$.
\end{proof}

Da $T(G)$ immer eine direkte Summe von endlichen abelschen $p$-Gruppen ist, möchten wir diese klassifizieren. Das Hauptresultat ist:

\begin{thm}
Es sei $p$ eine Primzahl und $n\in \mathbb{N}$. 
Jede abelsche Gruppe der Ordnung $p^n$ ist eine direkte Summe $G=H_{1}\oplus \dots\oplus H_{k}$, wobei $H_{i}$ eine zyklische Untergruppe von $G$ ist mit $|H_{i}|=p^{m_{i}}$ für jedes $i=1,\dots,k$.
Die Zahlen $m_1,m_2,\dots,m_k$ sind eindeutig durch $G$ bestimmt. O.B.d.A. ordnen wir $m_{1}\geq m_{2}\geq \dots\geq m_{k}\geq 0$.
\end{thm}

\begin{nb}
Es gilt: $m_{1}+\dots+m_{k}=n$.
\end{nb}

\begin{lemma}
Es sei $G$ eine abelsche $p$-Gruppe und $g\in G$ ein Element maximaler Ordnung. 
Dann ist $\langle g\rangle$ ein direkter Summand von $G$ ($G=\langle g\rangle\oplus H, H\subset G$ geeignet).
\end{lemma}

\begin{proof}
Es sei $p^{\alpha}$ die maximale Ordnung eines Elements in $G$ und $g\in G$ mit $\ord(g)=p^{\alpha}$. O.B.d.A. $\langle g\rangle \neq G$ 
\begin{enumerate}
\item Schritt: Wir möchten zeigen, dass es eine Untergruppe $H\subseteq G$ gibt mit $|H|=p, H\cap \langle g\rangle =\{ 0\}$.

Es sei $h \in G,h\notin \langle g \rangle$ beliebig.
Es sei $\ord(h+\langle g\rangle)=p^{\beta}$ die Ordnung von $h+\langle p\rangle$ in der $p$-Gruppe $G/\langle g\rangle$.
Wegen $h\notin \langle g\rangle$ ist $\beta\geq1$. 
Da $p^{\beta}(h+\langle g\rangle)=\langle g\rangle$ ist, gilt $p^{\beta}h=lg$ mit geeigneten $l\in \mathbb{Z}$. 
Da $p^{\alpha}$ die maximale Ordnung eines Elements von $G$ ist, ist $\ord(h)$ ein Teiler von $p^{\alpha}$ und somit gilt:
\[ 0=p^{\alpha}h=p^{\alpha -\beta}(p^{\beta}h)=p^{\alpha -\beta}lg\]
Daraus folgt: $ \ord(g)=p^{\alpha} \divides p^{\alpha -\beta}l$ $(\beta \geq 1)$. 
Was nur möglich ist, wenn $p \divides l$. 
Wir schreiben $l=\lambda p, \lambda \in \mathbb{Z}$ und definieren: $h':=p^{\beta -1}h-\lambda g \in G - \langle g \rangle.$ Man rechnet $ ph' = p^{\beta}h - p \lambda g =lg-lg=0$.

$\ord(h')$ ist ein Teiler von $p$ und wegen $h'\neq 0$ muss $\ord(h')=p$ gelten. 
Die Untergruppe $H=\langle h'\rangle$ erfüllt $\langle g\rangle \cap H=\{ 0\}$.
\item Schritt: Wir beweisen die Aussage durch Induktion nach $|G|$.

Es sei $g\in G$ ein Element maximaler Ordnung und $H\subseteq G$ wie im 1. Schritt. 
Wir betrachten die natürliche Projektion $\pi :G\rightarrow G/H$. 
$G/H$ ist eine abelsche $p$-Gruppe mit $|G/H|<|G|$ und $\pi (g)\in G/H$ ist ein Element maximaler Ordnung.
($\pi$ ist surjektiv; $\ord(\pi (g))$ ist die kleinste Zahl $m\geq 1$ mit $mg\in H$)

Da die maximale Ordnung in $G/H$ höchstens $\ord(g)=p^{\alpha}$ ist, reicht es zu zeigen, dass $\ord(\pi (g))=p^{\alpha}$ gilt.
$m=\ord(\pi (g))$ ist die kleinste positive Zahl, sodass $mg\in H$. Aus $\langle g\rangle \cap H=\{ 0\}$ folgt:
\[ mg\in \langle g\rangle \cap H=\{ 0\} \Rightarrow mg=0 \Rightarrow p^{\alpha}=\ord(g) \divides m\]
Ferner gilt:
\[ p^{\alpha}\pi (g)=\pi (p^{\alpha}g)=\pi(0) =0\Rightarrow m \divides p^{\alpha}\]
Also gilt $m=p^{\alpha}$. Aus der Induktionsvoraussetzung folgt $G/H=\langle \pi (g)\rangle \oplus V'$. 
Wir setzen $V := \pi ^{-1}(V') \subseteq G$. Wir möchten zeigen $G=\langle g\rangle \oplus V$. Es sei $a\in \mathbb{Z}$, sodass $ag\in V$. Dann gilt:
\[ \pi (ag)\in \langle \pi (g)\rangle\ \cap V'=\{ 0\}\Rightarrow ag\in H\Rightarrow ag=0\Rightarrow \langle g\rangle \cap V=\{ 0\}.\]
Sei $a\in G$. Dann gilt:
\[ \pi (a)=c\pi (g)+\pi (v),c\in \mathbb{Z}, v\in V.\]
Somit folgt:
\[ \pi (a-cg-v )=0\in G/H\Rightarrow a=cg+v+h\Rightarrow G = \langle g\rangle \oplus V.\qedhere \]
\end{enumerate} 
\end{proof}

% VL 14.07.
\begin{proof}[Beweis des Satzes]
	Es sei $G$ abelsch, $p$ prim, $n\in\mathbb{N}$ mit $|G|=p^n$. Da $G$ endlich ist, gibt es ein $g_1\in G$ maximaler Ordnung, mit $\ord(g_1)=p^{m_1}$. Aus dem Lemma folgt
	\[ G=\langle g_1\rangle\oplus V_1 \]
	mit $|V_1|=\frac{|G|}{p^{m_1}}=p^{n-m_1}$. Induktiv definiert man für jedes $n\in\mathbb{N},j\geq 1$:
	\begin{itemize}
		\item $g_{j+1}\in V_j$ ein Element maximaler Ordnung in $V_j$,
		\item $m_{j+1}\in\mathbb{N}$, sodass $\ord(g_{j+1})=m_{j+1}$,
		\item $V_{j+1}\subset V_j$, die $V_j=\langle g_{j+1}\rangle\oplus V_{j+1}$ erfüllt.
	\end{itemize}
	Da $|G|$ endlich ist, existiert ein (minimaler) Index $k$, sodass $V_{k+1}=\{0\}$ gilt. Das heißt für $k$ gilt $V_k\neq \{0\},\; V_{k+1}=\{0\}$. Dann gilt weiterhin
	\[ 
        G 
        = \langle g_1\rangle\oplus V_1 
        = \langle g_1\rangle\oplus\langle g_2\rangle\oplus V_2 
        = \dots 
        = \langle g_1\rangle\oplus\langle g_2\rangle\oplus\dots\oplus\langle g_k\rangle\oplus \underbrace{V_{k+1}}_{=\{0\}}.\]
	Wenn man $H_j:=\langle g_j\rangle$ definiert, gilt also
	\[ G = H_1\oplus\dots\oplus H_j\oplus\dots\oplus H_k,\]
	wobei $H_j$ zyklisch der Ordnung $p^{m_j}$ ist und $m_1\geq m_2\geq\dots\geq m_k$. 
        
        Wir müssen noch zeigen, dass die $m_j$ eindeutig bestimmt sind ($g_j$ und $V_j$ sind \emph{nicht} eindeutig). Für eine abelsche Gruppe $A$ und $s\in\mathbb{Z},\; s\geq 1$, setzen wir $A(s) := \{x\in A:sx=0\}$. $A(s)$ ist eine Untergruppe, denn $A(s) = \ker(A \overset{\mu_s}{\to} A),\;\mu_s(a):=sa$ und $\mu_s$ ist für abelsche Gruppen immer ein Gruppenhomomorphismus. Ferner induziert jeder Homomorphismus $A\to B$ von abelschen Gruppen einen Homomorphismus $A(s)\to B(s)$. Für nicht abelsche Gruppen ist die Abbildung schon für $s=2$ kein Homomorphismus. Im Fall von $G$ erhalten wir für $s=p$:
	\[ G(p) = H_1(p)\oplus H_2(p)\oplus\dots\oplus H_k(p).\]
	Da $H_j$ immer zyklisch von Ordnung $p^{m_j}$, $m_j\geq 1$, ist, ist $H_j(p)$ eine zyklische Gruppe der Ordnung $p$. Daraus folgt $|G(p)|=p^k$. Dies zeigt, dass $k$ durch $G$ eindeutig bestimmt wird. Für $s=p^{\alpha}$, $\alpha\geq 1$ beliebig, erhalten wir
	\begin{itemize}
		\item für eine zyklische Gruppe $C$ der Ordnung $p^{\beta}$ gilt
			\[ C(p^{\alpha})\cong \begin{dcases}
 				\nicefrac{\mathbb{Z}}{p^{\alpha}\mathbb{Z}}\text{ falls $\beta\geq\alpha$,} \\
 				C\text{ falls $\beta\leq\alpha$.}
 \end{dcases}\]
		\item für $G$ ergibt dies
			\[G(p^{\alpha})=H_1(p^{\alpha})\oplus\dots\oplus H_k(p^{\alpha}).\]
	\end{itemize}
	Es sei $k'$ der größte Index, sodass $m_{k'}\geq\alpha$ gilt. Daraus folgt
	\[|G(p^{\alpha})|=p^{\alpha k'}\cdot p^{m_{k'+1}}\cdot\dots\cdot p^{m_k} = p^{\alpha k'+m_{k'+1}+\dots+m_k}.\]
	Mithilfe der obigen Formel werden $m_1,\dots,m_k$ von endlich vielen Werten von $|G(p^{\alpha})|$ eindeutig bestimmt.
	
	Zum Verständnis dieser Formel und der Eindeutigkeit: zum Beispiel gilt $|G(p^{\alpha})|=|G|=p^n$ für $\alpha\geq m_1$ und somit ist der Wert von $m_1$ bestimmt. Sind $m_1=m_2=\dots m_{j_1}>m_{j_1+1}\geq\dots\geq m_k$, so gilt $|G(p^{m_1-1})|=p^{(m_1-1)j_1+m_{j_1+1}+\dots+m_k}=p^{(m_1-1)j_1+n-j_1m_1} = p^{n-j_1}$. Damit hat man den größten Exponent $j_1$ mit $m_{j_1}=1$ bestimmt.
\end{proof}

\begin{kor}
	Eine endlich erzeugte abelsche Gruppe $G$ ist immer eine direkte Summe endlich vieler zyklischer Gruppen, deren Ordnung unendlich oder Potenz einer Primzahl ist.
	
	Es existieren eine endlich bestimmte Zahl $r\geq 0$ und eindeutig bestimmte Primpotenzen $q_1,\dots q_l$, sodass gilt:
	\[ G\cong F\oplus T(G), \]
wobei $F\cong \nicefrac{G}{T(G)}$ frei abelsch vom Rang $r$ ist und 
	\[ T(G) \cong \nicefrac{\Z}{q_1\Z}\oplus\nicefrac{\Z}{q_2\Z}\oplus\dots\oplus\nicefrac{\Z}{q_l\Z}.\]
\end{kor}

\begin{proof}
	Die Gruppe $F:=\nicefrac{G}{T(G)}$ ist frei abelsch. Also ist $G\cong F\oplus T(G)$. Da $G$ endlich erzeugt ist, gilt $\rank(F)=r<\infty$. $T(G)$ ist eine endlich erzeugbare abelsche Torsionsgruppe. Also ist $T(G)$ endlich. Wir betrachten die Zerlegung von $T(G)$ in $p$-primäre Komponenten
	\[ T(G) = G_{p_1}\oplus\dots\oplus G_{p_t}, \]
	wobei $p_1,\dots,p_t$ die Primfaktoren von $|T(G)|=p_1^{e_1}\dots p_t^{e_t}$ sind.
	
	Wdh.: $p$ Primzahl, $G_p := \{x\in G: p^nx=0\text{ für ein $n\geq 0$}\}\subset T(G)$. Insbesondere gilt: $\left(T(G)\right)_p=G_p$. $G_p$ ist \emph{die} $p$-Sylowuntergruppe von $T(G)$.
	
	Für $p<p_1,\dots,p_t$ folgt aus dem letzten Satz
		\[ G_p \cong \nicefrac{\Z}{p^{m_1}\Z}\oplus\dots\oplus\nicefrac{\Z}{p^{m_k}\Z}.\qedhere\]
\end{proof}

\begin{ex}
	Wir bestimmen bis auf Isomorphie alle abelschen Gruppen der Ordnung $1350$. Die Gruppen sind endlich, also ist der Rang für die Zerlegung $r=0$. Die Primfaktorzerlegung ist $1350=2\cdot 3^3\cdot 5^2$, also ist eine Gruppe $G$ der Ordnung $1350$ gerade $G = G_2\oplus G_3\oplus G_5$. Wir erhalten $|G_2|=2$, also $G_2\cong\nicefrac{\Z}{2\Z}$ zyklisch. Außerdem $|G_3|=3^3=27$, somit suchen wir $k\geq 1$ und $m_1\geq m_2\geq\dots\geq m_k>0$ mit $m_1+\dots +m_k=3$. Mögliche Zerteilungen sind $k=1,m_1=3$: $G_3\cong\nicefrac{\Z}{27\Z}$ zyklisch,
	 $k=2,m_1=2,m_2=1$: $G_3\cong\nicefrac{\Z}{9\Z}\oplus\nicefrac{\Z}{3\Z}$,
	 $k=3,m_1=m_2=m_3=1$: $G_3\cong\nicefrac{\Z}{3\Z}\oplus\nicefrac{\Z}{3\Z}\oplus\nicefrac{\Z}{3\Z}$. Für $|G_5|=25$ erhalten wir die beiden Möglichkeiten $k=1, G_5\cong\nicefrac{\Z}{25\Z}$, $k=2,G_5\cong\nicefrac{\Z}{5\Z}\oplus\nicefrac{\Z}{5\Z}$. Bis auf Isomorphie gibt es also sechs abelsche Gruppen der Ordnung $1350$.
\end{ex}

\begin{ex}
	Abelsche Gruppen der Ordnung $3^6$. Für die Zerlegung ist $1\leq k\leq 6$ und wir erhalten
	$k=1,m_1=6$, $G\cong\nicefrac{\Z}{3^6\Z}$,
	$k=2,m_1\geq m_2=6-m_1$ für $m_1\in\{3,4,5\}$, $G\cong\nicefrac{\Z}{3^5\Z}\oplus\nicefrac{\Z}{3\Z}$, $G\cong\nicefrac{\Z}{3^4\Z}\oplus\nicefrac{\Z}{3^2\Z}$, $G\cong\nicefrac{\Z}{3^3\Z}\oplus\nicefrac{\Z}{3^3\Z}$,
	$k=3,(m_1,m_2,m_3)\in\{(4,1,1),(3,2,1),(2,2,2)\}$, $G\cong\nicefrac{\Z}{3^4\Z}\oplus\nicefrac{\Z}{3\Z}\oplus\nicefrac{\Z}{3\Z}$, $G\cong\nicefrac{\Z}{3^3\Z}\oplus\nicefrac{\Z}{3^2\Z}\oplus\nicefrac{\Z}{3\Z}$, $G\cong\nicefrac{\Z}{3^2\Z}\oplus\nicefrac{\Z}{3^2\Z}\oplus\nicefrac{\Z}{3^2\Z}$, und so weiter %oder in Potenzschreibweise
	%$k=4,m_1=6$, $G\cong\nicefrac{\Z}{3^6\Z}$,
	%$k=5,m_1=6$, $G\cong\nicefrac{\Z}{3^6\Z}$,
	%$k=6,m_1=6$, $G\cong\nicefrac{\Z}{3^6\Z}$.
\end{ex}


\end{document}
